\thispagestyle{newchap}
\begin{center}
  \large\textbf{\engtatitle{}}
\end{center}

\addcontentsline{toc}{chapter}{ABSTRACT}

\vspace{2ex}

\begingroup
% Menghilangkan padding
\setlength{\tabcolsep}{0pt}

\noindent
\begin{tabularx}{\textwidth}{l >{\centering}m{3em} X}
  \emph{Name / NRP}     & : & \name{} / \nrp{}         \\

  \emph{Department}    & : & \engdepartment FTIRS-ITS  \\

  \emph{Advisor} & : & \advisor{}   \\
  
\end{tabularx}
\endgroup

\vspace{2ex}

\begingroup
\noindent
\begin{large}
	\textit{\textbf{Abstract}}
\end{large}
\endgroup

% Ubah paragraf berikut dengan abstrak dari tugas akhir dalam Bahasa Inggris
\emph{Helicopters in motion have complex and intricate components. Damage can occur at any time to any part of the helicopter. One such damage that occurs on helicopters is ground resonance. Testing of the AS565 MBe Panther helicopter conducted by PTDI only uses 6 sensors at 3 test points, while in general ground vibration tests use as many as 300 to 500 sensors. So a study is needed to analyze using mathematical and simulation approaches that are validated using measurement data. Measurements were made using FTIS (Flight Test Instrumentation System) and accelerometer sensors with varying pressure conditions on the landing gear. Furthermore, the measurement data is processed using analysis with the help of Matlab computing and simulation in Femap (Finite Element Modeling And Postprocessing) software. From the use of 6 sensors, the dominant frequency error for mathematics and simulation is 2.63$\%$ and 42.5$\%$ respectively, then from the root mean square error (RMSE) is 1.7$\%$ and 3.40$\%$ respectively. The data processing results of the FTIS results obtained damping ratio quantification of 0.0542, 0.0479, 0.0433, and 0.0505 respectively at the rate of roll, rate of pitch, rate of yaw and acceleration in the y-axis direction. After the oscillation cycle limit is obtained, the percentage of frequency response to the oscillation cycle limit is found for $f_1$ (5.92-6.08Hz) by 0$\%$, $f_2$ (23.68-24.32Hz) by 15.88$\%$, $f_3$ (47.36-48.64Hz) by 4.27$\%$ and $f_4$ (71.04-72.96Hz) by 0.62$\%$. In the modification results, mathematically obtained an additional percentage of 2$\%$ against the frequency response of the chopper without modification. Then in simulation modeling using the stick model form, the frequency in the 6th mode that correlates with the ground resonance phenomenon is 4.31Hz with a response percentage of 3.40$\%$. So that after a series of mathematical approach results and simulation of normal conditions and modifications are carried out, no potential ground resonance is found on the AS565 MBe Panther helicopter.}

\vspace{2ex}

\begingroup
\noindent
\begin{normalsize}
\textbf{\emph{Keywords}: \emph{Frequency}, \emph{Vibration}, \emph{Helicopter}, \emph{Resonance}, \emph{Response}.}
\end{normalsize}
\endgroup


