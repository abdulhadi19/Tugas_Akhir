\begin{center}
  \large\textbf{ABSTRACT}
\end{center}

\addcontentsline{toc}{chapter}{ABSTRACT}

\vspace{2ex}

\begingroup
% Menghilangkan padding
\setlength{\tabcolsep}{0pt}

\noindent
\begin{tabularx}{\textwidth}{l >{\centering}m{3em} X}
  \emph{Name}     & : & \name{}         \\

  \emph{Title}    & : & \engtatitle{}   \\

  \emph{Advisors} & : & \advisor{}   \\
  
\end{tabularx}
\endgroup

% Ubah paragraf berikut dengan abstrak dari tugas akhir dalam Bahasa Inggris
\emph{Helicopters in motion have complex and intricate components. Damage can occur at any time to any part of the helicopter. One such damage that occurs in helicopters is ground resonance where the regressive mode merges with the motion mode of the helicopter frame (fuselage). In this study, three stages of work were carried out to identify the phenomenon of ground resonance in helicopters. That is through measurement, calculation and simulation modeling on Femap. In this final project, it has been explained about the potential occurrence of ground resonance in the modified AS 565 MBe Panther helicopter through ground test data, it has also analyzed the limit of its oscillation cycle, as well as the frequency response of the modified AS 565 MBe Panther helicopter to ground tests and mathematical calculations. Then in the last section, modeling has been made for simulation using the finite element method in the Femap software for the ground resonance phenomenon. At the end of the analysis, no potential ground resonance was found in the results of measurement, calculation and simulation data.}

% Ubah kata-kata berikut dengan kata kunci dari tugas akhir dalam Bahasa Inggris
\emph{Keywords}: \emph{Frequency}, \emph{Vibration}, \emph{Helicopter}, \emph{Resonance}, \emph{Response}.
