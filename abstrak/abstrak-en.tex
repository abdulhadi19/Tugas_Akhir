\begin{center}
  \large\textbf{ABSTRACT}
\end{center}

\addcontentsline{toc}{chapter}{ABSTRACT}

\vspace{2ex}

\begingroup
% Menghilangkan padding
\setlength{\tabcolsep}{0pt}

\noindent
\begin{tabularx}{\textwidth}{l >{\centering}m{3em} X}
  \emph{Name}     & : & \name{}         \\

  \emph{Title}    & : & \engtatitle{}   \\

  \emph{Advisors} & : & \advisor{}   \\
  
\end{tabularx}
\endgroup

% Ubah paragraf berikut dengan abstrak dari tugas akhir dalam Bahasa Inggris
\emph{The helicopter in motion has complex and intricate components. Damage can occur at any time to parts of the helicopter. One such damage that occurs in helicopters is ground resonance where the regressive mode merges with the motion mode of the helicopter frame (fuselage). In this study, 3 stages of work were carried out to identify the phenomenon of ground resonance in helicopters. Namely through measurement, per calculation and simulation modeling on FEMAP. This final project has explained the potential for ground resonance in the modification of the AS 565 MBe Panther helicopter through ground test data, has also analyzed the limits of its oscillation cycle, and the frequency response of the modified AS 565 MBe Panther helicopter to the ground test and mathematical calculations. Then in the last section, modeling has been made for the simulation using the finite element method in the FEMAP software for the ground resonance phenomenon.}

% Ubah kata-kata berikut dengan kata kunci dari tugas akhir dalam Bahasa Inggris
\emph{Keywords}: \emph{Frequency}, \emph{Vibration}, \emph{Helicopter}, \emph{Resonance}, \emph{Response}.
