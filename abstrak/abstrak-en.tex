\thispagestyle{newchap}
\begin{center}
  \large\textbf{ABSTRACT}
\end{center}

\addcontentsline{toc}{chapter}{ABSTRACT}

\vspace{2ex}

\begingroup
% Menghilangkan padding
\setlength{\tabcolsep}{0pt}

\noindent
\begin{tabularx}{\textwidth}{l >{\centering}m{3em} X}
  \emph{Name}     & : & \name{}         \\

  \emph{Title}    & : & \engtatitle{}   \\

  \emph{Advisors} & : & \advisor{}   \\
  
\end{tabularx}
\endgroup

% Ubah paragraf berikut dengan abstrak dari tugas akhir dalam Bahasa Inggris
\emph{Helicopters in motion have complex and intricate components. Damage can occur at any time to any part of the helicopter. One such damage that occurs on helicopters is ground resonance. Testing of the AS565 MBe Panther helicopter conducted by PTDI only uses 6 sensors at 3 test points, while in general ground tests use as many as 300 to 500 sensors. So a study is needed to analyze using mathematical and simulation approaches that are validated using measurement data. The dominant frequency errors for mathematics and simulation are 2.63$\%$ and 42.5$\%$ respectively and the root mean square errors are 1.7$\%$ and 3.40$\%$. There is no potential ground resonance on the AS565 MBe Panther helicopter and its modifications through ground test data with damping ratio quantification of 0.0542, 0.0479, 0.0433, and 0.0505 respectively at the rate of roll, rate of pitch, rate of yaw and y-axis acceleration. The frequency response of the oscillation cycle limit at $f_1$ is 0$\%$, $f_2$ is 15.88$\%$, $f_3$ is 4.27$\%$ and $f_4$ is 0.62$\%$. In the modified results, mathematically obtained an additional 2$\%$ of the helicopter frequency response. The results of simulation modeling using the stick model form against the ground resonance phenomenon are obtained at a frequency of 4.31 Hz (6th mode) with a response of 3.40$\%$.}

\emph{Keywords}: \emph{Frequency}, \emph{Vibration}, \emph{Helicopter}, \emph{Resonance}, \emph{Response}.
