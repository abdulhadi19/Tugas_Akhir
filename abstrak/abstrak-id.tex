\thispagestyle{newchap}
\begin{center}
  \large\textbf{\tatitle{}}
\end{center}

\addcontentsline{toc}{chapter}{ABSTRAK}

\vspace{2ex}

\begingroup
% Menghilangkan padding
\setlength{\tabcolsep}{0pt}

\noindent
\begin{tabularx}{\textwidth}{l >{\centering}m{2em} X}
  Nama Mahasiswa / NRP   & : & \name{} / \nrp{}        \\

  Departemen 	 		 & : & \department{} FTIRS-ITS \\

  Dosen Pembimbing       & : & \advisor{}   \\
  
\end{tabularx}
\endgroup

\vspace{2ex}

\begingroup
\noindent
\begin{large}
	\textbf{Abstrak}
\end{large}
\endgroup

% Ubah paragraf berikut dengan abstrak dari tugas akhir
Helikopter dalam gerakannya memiliki komponen yang kompleks dan rumit. Kerusakan dapat terjadi sewaktu-waktu pada bagian helikopter. Salah satu kerusakan tersebut yang terjadi pada helikopter adalah \textit{ground resonance}. Pengujian terhadap helikopter AS565 MBe Panther yang dilakukan oleh PTDI hanya menggunakan 6 sensor pada 3 titik pengujian, sedangkan pada umumnya \textit{ground vibration test} menggunakan sebanyak 300 hingga 500 sensor. Sehingga diperlukan suatu studi untuk menganalisis menggunakan pendekatan matematis dan simulasi yang divalidasi menggunakan data pengukuran. Pengukuran dilakukan menggunakan FTIS (\textit{Flight Test Instrumentation System}) dan sensor akselerometer dengan variasi kondisi tekanan pada \textit{landing gear}. Selanjutnya data pengukuran diolah menggunakan analisis dengan bantuan komputasi Matlab dan simulasi pada \textit{software} Femap (\textit{Finite Element Modeling And Postprocessing}). Dari penggunaan 6 sensor didapatkan \textit{error} dominan frekuensi untuk matematis dan simulasi berturut-turut sebesar 2.63$\%$ dan 42.5$\%$, kemudian dari \textit{root mean square error} (RMSE) nya berturut-turut sebesar 1.7$\%$ dan 3.40$\%$. Hasil pengolahan data hasil FTIS didapatkan kuantifikasi \textit{damping ratio} sebesar 0.0542, 0.0479, 0.0433, dan 0.0505 masing-masing pada \textit{rate of roll}, \textit{rate of pitch}, \textit{rate of yaw} dan percepatan arah sumbu-y. Setelah batas siklus osilasi didapatkan, ditemukan persentase respon frekuensi terhadap batas siklus osilasi tersebut untuk $f_1$ (5.92-6.08Hz) sebesar 0$\%$, $f_2$ (23.68-24.32Hz) sebesar 15.88$\%$, $f_3$ (47.36-48.64Hz) sebesar 4.27$\%$ dan $f_4$ (71.04-72.96Hz) sebesar 0.62$\%$. Pada hasil modifikasi, secara matematis didapatkan persentase tambahan sebanyak 2$\%$ terhadap respon frekuensi helikopter tanpa modifikasi. Kemudian pada pemodelan simulasi menggunakan bentuk stik model didapatkan frekuensi pada mode ke-6 yang berkorelasi terhadap fenomena \textit{ground resonance} yaitu 4.31Hz dengan persentase respon sebanyak 3.40$\%$. Sehingga setelah serangkaian hasil pendekatan matematis dan simulasi kondisi normal serta modifikasinya dilakukan, tidak ditemukan potensi \textit{ground resonance} pada helikopter AS565 MBe Panther.

\vspace{2ex}
% Ubah kata-kata berikut dengan kata kunci dari tugas akhir
\begingroup
\noindent
\begin{normalsize}
\textbf{Kata Kunci: Frekuensi, Getaran, Helikopter, \textit{Resonance}, Respon.}
\end{normalsize}
\endgroup

