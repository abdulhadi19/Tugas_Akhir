\begin{center}
  \large\textbf{ABSTRAK}
\end{center}

\addcontentsline{toc}{chapter}{ABSTRAK}

\vspace{2ex}

\begingroup
% Menghilangkan padding
\setlength{\tabcolsep}{0pt}

\noindent
\begin{tabularx}{\textwidth}{l >{\centering}m{2em} X}
  Nama Mahasiswa    & : & \name{}         \\

  Judul Tugas Akhir & : & \tatitle{}      \\

  Pembimbing        & : & \advisor{}   \\
  
\end{tabularx}
\endgroup

% Ubah paragraf berikut dengan abstrak dari tugas akhir
Helikopter dalam gerakannya memiliki komponen yang kompleks dan rumit. Kerusakan dapat terjadi sewaktu-waktu pada bagian helikopter. Salah satu kerusakan tersebut yang terjadi pada helikopter adalah \textit{ground resonance} dimana mode regresif melebur dengan mode gerakan kerangka helikopter (\textit{fuselage}). Dalam penelitian ini dilakukan 3 tahap pengerjaan untuk mengindentifikasi fenomena \textit{ground resonance} pada helikopter. Yaitu melalui pengukuran, perhitungan dan pemodelan simulasi pada Femap yang telah divalidasi dalam pengerjaan tugas akhir ini. Tidak ditemukan adanya potensi terjadinya \textit{ground resonance} pada modifikasi helikopter AS 565 MBe Panther melalui data hasil \textit{ground test} dengan kuantifikasi \textit{damping ratio} sebesar 0.0542, 0.0479, 0.0433, dan 0.0505 masing-masing pada \textit{rate of roll}, \textit{rate of pitch}, \textit{rate of yaw} dan percepatan arah sumbu-y, telah analisis juga mengenai respon dari batas siklus osilasinya pada $f_1$ sebesar 0$\%$, $f_2$ sebesar 15.88$\%$, $f_3$ sebesar 4.27$\%$ dan $f_4$ sebesar 0.62$\%$. serta respon frekuensi yang dimiliki Helikopter AS 565 MBe Panther hasil modifikasi terhadap \textit{ground test} dan perhitungan matematis. Pada hasil modifikasi didapatkan sebanyak 2$\%$ dari respon frekuensi helikopter, sehingga didapatkan akan terdapat tambahan potensi sebesar 2$\%$ dari hasil modifikasi pada helikopter. Kemudian pada bagian terakhir, ditemukan pemodelan simulasi menggunakan \textit{finite element method} pada \textit{software} Femap dengan frekuensi yang berkorelasi terhadap fenomena \textit{ground resonance} yaitu pada frekuensi 4.31 Hz (mode ke-6) dengan respon sebanyak 3$\%$.

% Ubah kata-kata berikut dengan kata kunci dari tugas akhir
Kata Kunci: Frekuensi, Getaran, Helikopter, \textit{Resonance}, Respon.