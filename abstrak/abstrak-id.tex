\thispagestyle{newchap}
\begin{center}
  \large\textbf{ABSTRAK}
\end{center}

\addcontentsline{toc}{chapter}{ABSTRAK}

\vspace{2ex}

\begingroup
% Menghilangkan padding
\setlength{\tabcolsep}{0pt}

\noindent
\begin{tabularx}{\textwidth}{l >{\centering}m{2em} X}
  Nama Mahasiswa    & : & \name{}         \\

  Judul Tugas Akhir & : & \tatitle{}      \\

  Pembimbing        & : & \advisor{}   \\
  
\end{tabularx}
\endgroup

% Ubah paragraf berikut dengan abstrak dari tugas akhir
Helikopter dalam gerakannya memiliki komponen yang kompleks dan rumit. Kerusakan dapat terjadi sewaktu-waktu pada bagian helikopter. Salah satu kerusakan tersebut yang terjadi pada helikopter adalah \textit{ground resonance}. Pengujian terhadap helikopter AS565 MBe Panther yang dilakukan oleh PTDI hanya menggunakan 6 sensor pada 3 titik pengujian, sedangkan pada umumnya \textit{ground test} menggunakan sebanyak 300 hingga 500 sensor. Sehingga diperlukan suatu studi untuk menganalisis menggunakan pendekatan matematis dan simulasi yang divalidasi menggunakan data pengukuran. Didapatkan error dominan frekuensi untuk matematis dan simulasi masing-masing sebesar 2.63$\%$ dan 42.5$\%$ serta \textit{root mean square error} 1.7$\%$ dan 3.40$\%$. Tidak terdapat potensi \textit{ground resonance} pada helikopter AS565 MBe Panther beserta modifikasinya melalui data hasil \textit{ground test} dengan kuantifikasi \textit{damping ratio} sebesar 0.0542, 0.0479, 0.0433, dan 0.0505 masing-masing pada \textit{rate of roll}, \textit{rate of pitch}, \textit{rate of yaw} dan percepatan arah sumbu-y. Didapatkan respon frekuensi dari batas siklus osilasi pada $f_1$ sebesar 0$\%$, $f_2$ sebesar 15.88$\%$, $f_3$ sebesar 4.27$\%$ dan $f_4$ sebesar 0.62$\%$. Pada hasil modifikasi, secara matematis didapatkan tambahan sebanyak 2$\%$ dari respon frekuensi helikopter. Didapatkan hasil pemodelan simulasi menggunakan bentuk stik model terhadap fenomena \textit{ground resonance} yaitu pada frekuensi 4.31 Hz (mode ke-6) dengan respon sebanyak 3.40$\%$.

% Ubah kata-kata berikut dengan kata kunci dari tugas akhir
Kata Kunci: Frekuensi, Getaran, Helikopter, \textit{Resonance}, Respon.