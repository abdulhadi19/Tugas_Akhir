\chapter{PENDAHULUAN}
\label{chap:pendahuluan}
% Ubah bagian-bagian berikut dengan isi dari pendahuluan

\section{Latar Belakang}
\label{sec:latarbelakang}

Helikopter merupakan \textit{rotorcraft} yang dalam gerakannya memiliki berbagai jenis getaran. Dari beberapa getaran yang dihasilkan, dapat menyebabkan terjadinya kerusakan secara instan pada helikopter. Hal itu dapat disebabkan oleh interaksi antara gerakan rotor yang terlambat (\textit{lagging motion}) terhadap salah satu \textit{mode-shape} dari gerakan kerangka helikopter \cite{Dzy1}. Secara khusus kerusakan tersebut terjadi pada fenomena \textit{ground resonance}.  Kondisi \textit{ground resonance} terjadi pada kecepatan tertentu oleh rotor, dimana mode regresif melebur dengan mode gerakan kerangka helikopter (\textit{fuselage}) yang mengakibatkan kerusakan langsung pada badan helikopter \cite{DASIL}. 

Coleman pada tahun 1957 menunjukkan bahwa \textit{ground resonance} merupakan bentuk ketidakstabilan yang tereksitasi pada helikopter dengan menggunakan analisis klasik. Pada dasarnya, beberapa gangguan dari luar akan memengaruhi rotor. Gangguan tersebut mengakibatkan pusat massa helikopter bergeser dan membentuk gaya inersia yang bekerja melawan badan helikopter \cite{Coleman}. Beberapa pengamatan menggunakan linearisasi persamaan gerak helikopter telah dijelaskan pada \cite{Friedmann} dan \cite{bielawa2006}. Akan tetapi, helikopter dengan redaman yang memiliki sistem elastomer (komponen yang meredam dan mengurangi getaran) mempunyai karakteristik respon yang nonlinier, kemudian dilakukan analisis secara nonlinier seperti pada \cite{KUNZ2002383} dan \cite{DASIL}. 

Helikopter terdiri dari sistem yang rumit dan kompleks. Kemampuan \textit{flap}, \textit{feather}, \textit{lead-lag} helikopter pada rotornya merupakan helikopter jenis \textit{fully-articulated rotor system} \cite{wagtendonk2006principles}. Helikopter jenis ini memiliki potensi yang besar untuk mengalami kondisi \textit{ground resonance} \cite{Eckert2007AnalyticalAA}. Salah satu helikopter jenis ini adalah helikopter AS 565 MBe Panther yang merupakan bagian helikopter AS 565 MBe versi angkatan laut. Helikopter ini dirancang untuk beroperasi dari geladak kapal, lokasi lepas pantai, dan lokasi berbasis darat \cite{AS565MBe}.

PT Dirgantara Indonesia atau yang dikenal dengan PTDI adalah salah satu perusahaan \textit{aerospace} di Asia dengan kompetensi inti dalam desain dan pengembangan pesawat, pembuatan struktur pesawat, produksi pesawat, dan layanan pesawat untuk sipil dan militer dari pesawat ringan dan menengah \cite{PTDI}. PTDI melakukan sebuah modifikasi terhadap helikopter AS 565 MBe Panther untuk suatu kebutuhan tertentu. Disisi lain, keamanan dan keselamatan penerbangan merupakan prioritas utama dalam dunia aviasi. Berdasarkan Peraturan Menteri Perhubungan nomor 8 tahun 2010 mengenai Program Keselamatan Penerbangan Nasional, diperlukan suatu usaha untuk mengidentifikasi bahaya dan proses manajemen risiko secara berkesinambungan \cite{MenPerhub}. Sehingga dalam proses modifikasi dibutuhkan serangkaian uji untuk memastikan hasil modifikasi yang dilakukan pada helikopter telah aman dan dapat digunakan sebagaimana tujuannya.

Pengujian yang dilakukan oleh PTDI menghasilkan parameter vibrasi yang dapat diidentifikasi melalui pengolahan data dan analisa matematika. Pengujian tersebut dilakukan di PTDI pada kondisi saat masih di tanah \textit{ground test} dengan menggunakan \textit{input} impuls \textit{lateral} dan \textit{longitudinal} pada \textit{control pilot} dengan variasi kondisi \textit{rotor speed}, \textit{stability augmentation system} (SAS) dan \textit{landing gear shock absorbers}. Dalam pengerjaan penelitian ini, akan ditambahkan aspek simulasi menggunakan \textit{software} FEMAP (\textit{Finite Element Modeling And Postprocessing}) yang dapat merepresentasikan hasil akhir dari tujuan tugas akhir ini. Aspek matematis yang digunakan merupakan perhitungan dari beberapa acuan yang dijadikan sebagai referensi. Serangkaian pengujian yang dilakukan diatas merupakan upaya untuk membuktikan bahwa tidak terdapat potensi terjadinya \textit{ground resonance} serta memberikan batasan pada osilasi rotor helikopter untuk menghindari kondisi \textit{ground resonance}.

\section{Rumusan Masalah}
\label{sec:rumusan masalah}

Dari permasalahan yang telah disebutkan pada latar belakang, berikut ini merupakan rumusan masalah yang terdapat pada tugas akhir ini:

\begin{enumerate}[nolistsep]
	
	\item Apakah potensi fenomena \textit{ground resonance} terjadi pada modifikasi helikopter AS 565 MBe Panther melalui pengujian di tanah (\textit{ground test})?
	
	\item Bagaimana batas siklus osilasi helikopter AS 565 MBe Panther hasil modifikasi terhadap fenomena \textit{ground resonance} menggunakan standar MIL-STD-810H-Method-514.8 dan pendekatan matematis?
	
	\item Bagaimana respon frekuensi helikopter AS 565 MBe Panther hasil modifikasi terhadap pengujian dan perhitungan matematis terhadap fenomena \textit{ground resonance}?
	
	\item Bagaimana pemodelan simulasi dari helikopter modifikasi AS 565 MBe Panther menggunakan \textit{finite element} pada \textit{software} FEMAP terhadap fenomena \textit{ground resonance}? 
\end{enumerate}

\section{Tujuan}
\label{sec:Tujuan}

\begin{enumerate}[nolistsep]

	\item Mengetahui potensi terjadinya \textit{ground resonance} pada modifikasi helikopter AS 565 MBe Panther dari pengujian yang dilakukan di tanah (\textit{ground test}).
	
	\item Mendapatkan batas siklus osilasi modifikasi helikopter AS 565 MBe Panther hasil modifikasi yang dilakukan oleh PTDI terhadap fenomena \textit{ground resonance} menggunakan standar MIL-STD-810H-Method-514.8 dan pendekatan matematis.
	
	\item Mengetahui dan menganalisis  respon frekuensi helikopter AS 565 MBe Panther hasil modifikasi terhadap pengujian dan perhitungan matematis terhadap fenomena \textit{ground resonance}.
	
	\item Mengetahui dan menganalisis pemodelan simulasi dari helikopter modifikasi AS 565 MBe Panther menggunakan \textit{finite element} pada \textit{software} FEMAP terhadap fenomena \textit{ground resonance}.
\end{enumerate}

\section{Batasan Masalah}
\label{sec:batasanmasalah}

Batasan masalah dari tugas akhir ini adalah sebagai berikut:

\begin{enumerate}[nolistsep]

  \item Lokasi tempat pengerjaan tugas akhir dilakukan di PT. Dirgantara Indonesia dan di Laboratorium Vibrasi dan Akustik Teknik Fisika ITS.

  \item Objek yang digunakan pada penelitian ini adalah helikopter AS 565 MBe Panther yang telah dimodifikasi oleh PT. Dirgantara Indonesia.
  
  \item Helikopter modifikasi yang dimaksud hanya terbatas pada penambahan massa helikopter secara homogen pada seluruh bagian helikopter.
  
  \item Material pada helikopter AS 565 MBe Panther menggunakan pendekatan pada sumber yang ada saja tanpa ada informasi langsung dari PTDI.

  \item Pengambilan data pengukuran helikopter dilakukan saat kondisi cerah pagi hari dengan batas waktu dari pukul 07.00 hingga 11.00 WIB.
  
  \item \textit{Software} yang digunakan dalam pengerjaan tugas akhir ini adalah Matlab dan FEMAP.
  
  \item Rentang waktu pengambilan data dilakukan dari bulan Juli-September 2022 dan Maret-Mei 2023.

\end{enumerate}

\section{Sistematika Laporan}
\label{sec:sistematikalaporan}

Laporan tugas akhir ini terdiri dari 5 BAB yang dijelaskan secara runtut untuk menggambarkan pengerjaan tugas akhir.

\begin{enumerate}[nolistsep]

  \item \textbf{BAB I Pendahuluan}

        Bab ini berisi latar belakang yang menggambarkan landasan utama mengapa tugas akhir ini dilakukan. Dari permasalahan terhadap fenomena \textit{ground resonance} didapatkan beberapa tujuan serta batasan dalam penyelesaiannya.

        \vspace{2ex}

  \item \textbf{BAB II Tinjauan Pustaka dan Dasar Teori}

        Bab ini berisi informasi terkait acuan, perhitungan, dan penjelasan dari terminologi vibrasi yang berkaitan dengan \textit{ground resonance} baik dari aspek prinsip, matematis, serta simulasi.

        \vspace{2ex}

  \item \textbf{BAB III Metodologi Penelitian}

        Bab ini berisi informasi detail terhadap pengambilan data yang dilakukan di PT. Dirgantara Indonesia, alur pengolahan data dan cara apa saja yang digunakan untuk mendapatkan variabel-variabel yang berkaitan dengan \textit{ground resonance}.

        \vspace{2ex}

  \item \textbf{BAB IV Hasil dan Pembahasan}

        Bab ini berisi informasi hasil pengukuran dan pengolahan data yang telah dilakukan sesuai pada BAB III. Pembahasan berupa analisis secara kuantitatif dijelaskan pada Bab ini sebagai bentuk interpretasi data yang didapat terhadap fenomena yang berkaitan dengan \textit{ground resonance}.

        \vspace{2ex}

  \item \textbf{BAB V Penutup}

        Bab ini berisi kesimpulan dan saran. Kesimpulan pada bab ini akan menjelaskan jawaban dari rumusan masalah yang dimiliki pada BAB I. Adapun saran pada bab ini merupakan hal-hal yang sekiranya perlu untuk dievaluasi dimasa mendatang.

\end{enumerate}
