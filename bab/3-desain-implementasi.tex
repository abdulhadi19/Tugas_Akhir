\chapter{METODOLOGI PENELITIAN}
\label{chap:metodologipenelitian}

% Ubah bagian-bagian berikut dengan isi dari desain dan implementasi
\thispagestyle{newchap}
Untuk mencapai tujuan dalam tugas akhir ini, dilakukan beberapa tahap yang dimulai dengan pemahaman fenomena \textit{ground resonance} pada helikopter melalui studi pustaka serta teori pendukung. Berikutnya adalah melakukan perumusan terhadap permasalahan yang akan diselesaikan, dalam hal ini adalah terkait dengan fenomena \textit{ground resonance} helikopter.

Terdapat 3 pengerjaan yang dilakukan secara paralel dan berurutan, yaitu pengukuran di tanah \textit{ground test} helikopter, perhitungan matematis secara teori dan komputasi Matlab serta simulasi menggunakan \textit{software} Femap. Kemudian dilakukan validasi hasil perhitungan dan simulasi terhadap hasil pengukuran \textit{ground test}. Sehingga apabila direpresentasikan dalam bentuk alur pengerjaannya dapat dilihat pada gambar \ref{fig:TA_flow-Page-1.jpg} dan \ref{fig:TA_flow-Page-2.jpg}.
 
\begin{figure}[h!]
	\centering
	\includegraphics[width=\textwidth]{gambar/TA_flow-Page-1.jpg}
	\caption{Diagram alir pengerjaan Tugas Akhir bagian 1.}
	\label{fig:TA_flow-Page-1.jpg}
\end{figure}

\begin{figure}[H]
	\centering
	\includegraphics[width=0.95\textwidth]{gambar/TA_flow-Page-2.jpg}
	\caption{Diagram alir pengerjaan Tugas Akhir bagian 2.}
	\label{fig:TA_flow-Page-2.jpg}
\end{figure}

\section{Studi Pustaka}
\label{sec:studipustaka}

Studi pustaka merupakan tahap untuk membaca dan memahami referensi solusi serta metode peneliti sebelumnya berkaitan dengan \textit{ground resonance} pada helikopter. Beberapa referensi yang mendukung terhadap topik ini adalah mengenai metode pengolahan data hasil pengukuran menggunakan MIL-STD-810H-Method-514.8 (vibrasi), perhitungan matematis yang menggunakan konsep dasar dari parameter pegas-peredam-massa, penyelesaian matematis dengan menggunakan lagrang, transformasi \textit{multiblade} matriks, dan solusi eigen dengan bantuan komputasi Matlab. Kemudian referensi lain berupa informasi untuk pemakaian \textit{software} Femap dalam aspek simulasinya.

\section{Pengambilan Data \textit{Ground Test}}
  \label{sec:pengukurandata}

Pengambilan data \textit{ground test} dilakukan dengan mengikuti mekanisme yang telah ditetapkan oleh PTDI. Terdapat 2 pengukuran yang dilakukan, pertama adalah pengukuran terhadap \textit{damping ratio} dari respon helikopter terhadap impuls yang diberikan oleh pilot. Kedua, adalah pengukuran yang didapatkan menggunakan sensor akselerometer yang dipasangkan di beberapa titik pada helikopter.

\subsection{Pengukuran Data Vibrasi pada FTIS}
Pengukuran yang didapatkan dari \textit{Flight Test Instrumentation System} (FTIS) merupakan pengukuran respon gerakan keseluruhan helikopter pada beberapa orientasi gerakan. Respon terhadap impuls lateral serta longitudinal oleh pilot digunakan untuk mengetahui redaman getaran pada struktur helikopter. Nilai \textit{logarithmic decrement} didapatkan dari hasil pengukuran oleh FTIS dengan melihat seberapa cepat amplitudo getaran helikopter teredam sepanjang waktu setelah diberi impuls oleh pilot, nilai ini nantinya akan digunakan untuk menghitung \textit{damping ratio} helikopter. FTIS terletak pada bagian pusat massa helikopter. Karena helikopter dianggap sebagai massa titik yang dapat bergerak pada orientasi 3 dimensi, baik secara getaran translasi ataupun getaran rotasi. Sehingga pada pengukuran ini, nantinya akan didapatkan \textit{output} getaran berupa \textit{roll, pitch, heading, rate of roll, rate of pitch, rate of yaw}, percepatan arah sumbu-x, y, dan z.

\subsection{Pengukuran Data Vibrasi Akselerometer}
\label{Pengukuran Data Vibrasi Akselerometer}

Pengukuran vibrasi dengan menggunakan akselerometer dilakukan dibeberapa titik pada helikopter. Konfigurasi mengenai peletakan sensor akselerometer dapat dilihat pada gambar \ref{peletakan_sensor.png} dan tabel \ref{tb:lokasiakselero}.

\begin{figure}[H]
	\centering
	\includegraphics[width=0.6\linewidth]{gambar/peletakan_sensor.png}
	\caption{Peletakan akseleromter pada Helikopter.}
	\label{peletakan_sensor.png}
\end{figure}

\begin{longtable}{|c|c|}
	\caption{Lokasi dan arah akseleromter.}
	\label{tb:lokasiakselero}                        	\\
	\hline
	\textbf{Channel} & \textbf{Lokasi akselerometer} 	\\
	\hline
	1            	 & Kursi pilot (20Z)             	\\
	\hline
	2			     & Bagian luar stik kopilot (16Z)   \\
	\hline
	3				 & Bagian luar stik kopilot	(18Y)   \\
	\hline
	4				 & Frame 9" (21Z)                   \\
	\hline
	5				 & Frame 9" (22X)					\\
	\hline
	6				 & Frame 9" (24Y)					\\
	\hline
\end{longtable}

Kemudian berikut (gambar \ref{tampak_atas.png}, \ref{tampak_depan.png}, \ref{tampak_samping.png}) merupakan skema Helikopter AS565 MBe Panther yang digunakan beserta keterangan dimensi dari beberapa arah skema.

\begin{figure}[H]
	\centering
	\includegraphics[width=0.65\linewidth]{gambar/tampak_samping.png}
	\caption{Skema penempatan \textit{channel} sensor dan dimensi helikopter AS565 MBe Panther (tampak samping).}
	\label{tampak_samping.png}
\end{figure}

\begin{figure}[H]
	\centering
	\includegraphics[width=0.6\linewidth]{gambar/tampak_depan.png}
	\caption{Skema penempatan \textit{channel} sensor dan dimensi helikopter AS565 MBe Panther (tampak depan).}
	\label{tampak_depan.png}
\end{figure}

\begin{figure}[H]
	\centering
	\includegraphics[width=0.7\linewidth]{gambar/tampak_atas.png}
	\caption{Skema penempatan \textit{channel} sensor dan dimensi helikopter AS565 MBe Panther (tampak atas).}
	\label{tampak_atas.png}
\end{figure}

Tabel \ref{tb:variasilanding} merupakan variasi kondisi pengujian pada \textit{landing gear absorbers}. Variasi tersebut dilakukan pada bagian ban dan oleo helikopter. Kuantifikasi variasi spesifik kondisi \textit{landing gear absorbers} pada tabel \ref{tb:variasilanding} diberikan pada tabel \ref{tb:propertiskuantitatif}. Sedangkan variasi \textit{nput} yang diberikan pada helikopter dapat dilihat pada tabel \ref{tb:variasi_input}. Sehingga secara teknis, dalam 1 kondisi pengujian terdapat 12 variasi \textit{input} yang diberikan kepada helikopter untuk kemudian selanjutnya akan diperhitungkan pada BAB \ref{chap:hasil dan pembahasan}.

\begin{table}[H]
	\centering
	\caption{Variasi Kondisi Pengukuran menggunakan FTIS.}
	\label{tb:variasilanding}
	\resizebox{0.7\textwidth}{!}{%
		\begin{tabular}{|c|ccccc|}
			\hline
			\multirow{3}{*}{Kondisi ke-} & \multicolumn{5}{c|}{Kondisi pengujian Oleo/Tyre} \\ \cline{2-6} 
			& \multicolumn{1}{c|}{Bagian Depan} & \multicolumn{2}{c|}{Bagian Kiri} & \multicolumn{2}{c|}{Bagian Kanan} \\ \cline{2-6} 
			& \multicolumn{1}{c|}{Oleo/Tyre} & \multicolumn{1}{c|}{Oleo} & \multicolumn{1}{c|}{Tyre} & \multicolumn{1}{c|}{Oleo} & Tyre \\ \hline
			1 & \multicolumn{1}{c|}{Nominal/Nominal} & \multicolumn{1}{c|}{Nominal} & \multicolumn{1}{c|}{Nominal} & \multicolumn{1}{c|}{Nominal} & Nominal \\ \hline
			2 & \multicolumn{1}{c|}{Nominal/Nominal} & \multicolumn{1}{c|}{High} & \multicolumn{1}{c|}{High} & \multicolumn{1}{c|}{Nominal} & Nominal \\ \hline
			3 & \multicolumn{1}{c|}{Nominal/Nominal} & \multicolumn{1}{c|}{Nominal} & \multicolumn{1}{c|}{Nominal} & \multicolumn{1}{c|}{High} & High \\ \hline
			4 & \multicolumn{1}{c|}{Nominal/Nominal} & \multicolumn{1}{c|}{Nominal} & \multicolumn{1}{c|}{Nominal} & \multicolumn{1}{c|}{Low} & Low \\ \hline
			5 & \multicolumn{1}{c|}{Nominal/Nominal} & \multicolumn{1}{c|}{Low} & \multicolumn{1}{c|}{Low} & \multicolumn{1}{c|}{Nominal} & Nominal \\ \hline
			6 & \multicolumn{1}{c|}{Nominal/Nominal} & \multicolumn{1}{c|}{High} & \multicolumn{1}{c|}{High} & \multicolumn{1}{c|}{High} & High \\ \hline
			7 & \multicolumn{1}{c|}{Nominal/Nominal} & \multicolumn{1}{c|}{Low} & \multicolumn{1}{c|}{Low} & \multicolumn{1}{c|}{Low} & Low \\ \hline
			8 & \multicolumn{1}{c|}{Nominal/Nominal} & \multicolumn{1}{c|}{High} & \multicolumn{1}{c|}{High} & \multicolumn{1}{c|}{Low} & Low \\ \hline
			9 & \multicolumn{1}{c|}{Nominal/Nominal} & \multicolumn{1}{c|}{Low} & \multicolumn{1}{c|}{Low} & \multicolumn{1}{c|}{High} & High \\ \hline
			10 & \multicolumn{1}{c|}{Nominal/Nominal} & \multicolumn{1}{c|}{High} & \multicolumn{1}{c|}{High} & \multicolumn{1}{c|}{Nominal} & High \\ \hline
			11 & \multicolumn{1}{c|}{Nominal/Nominal} & \multicolumn{1}{c|}{High} & \multicolumn{1}{c|}{High} & \multicolumn{1}{c|}{Low} & High \\ \hline
		\end{tabular}%
	}
\end{table}

\begin{table}[h]
	\centering
	\caption{Propertis kuantitatif variasi \textit{landing gear absorbers} pada ban dan oleo helikopter.}
	\label{tb:propertiskuantitatif}
	\resizebox{0.75\textwidth}{!}{%
		\begin{tabular}{|c|c|c|l|}
			\hline
			\multirow{3}{*}{NLG} & \multirow{2}{*}{Oleo} & \multirow{2}{*}{Nominal} & Hydraulic: 10 bar (145 psi) \\ \cline{4-4} 
			&  &  & Nitrogen 30$^o$C=42 bar(psi) \\ \cline{2-4} 
			& Tyre & Nominal & 5.5 bar (79.7 psi) \\ \hline
			\multirow{9}{*}{MLG} & \multirow{6}{*}{Oleo} & \multirow{2}{*}{Low} & Hydraulic: 10 bar (145 psi) \\ \cline{4-4} 
			&  &  & Nitrogen 15oC: HP=49.0 bar (711 psi) LP=4.0 bar (59.0 psi) \\ \cline{3-4} 
			&  & \multirow{2}{*}{Nominal} & Hydraulic: 10 bar (145 psi) \\ \cline{4-4} 
			&  &  & Nitrogen 30$^o$C: HP=51.5 bar (747 psi) LP=4.2 bar (61 psi) \\ \cline{3-4} 
			&  & \multirow{2}{*}{High} & Hydraulic: 10 bar (145 psi) \\ \cline{4-4} 
			&  &  & Nitrogen 45$^o$C: HP=54.0 bar (783 psi) LP=4.4 bar (63.82 psi) \\ \cline{2-4} 
			& \multirow{3}{*}{Tyre} & Low & 8.48 bar (123 psi) \\ \cline{3-4} 
			&  & Nominal & 10.8 bar (156.6 psi) \\ \cline{3-4} 
			&  & High & 13.1 bar (190 psi) \\ \hline
		\end{tabular}%
	}
\end{table}

\begin{table}[h]
	\centering
	\caption{Variasi \textit{input} pada \textit{ground test} Helikopter.}
	\label{tb:variasi_input}
	\resizebox{0.75\textwidth}{!}{%
		\begin{tabular}{|c|c|c|c|c|}
			\hline
			No & SAS & Power & Input Control & Name of Sequence \\ \hline
			1 & \multirow{6}{*}{OFF} & Ground Idle & Longitudinal & FILO \\ \cline{1-1} \cline{3-5} 
			2 &  & Ground Idle & Lateral & FILA \\ \cline{1-1} \cline{3-5} 
			3 &  & Flight Idle (on Ground) & Longitudinal & FFLO \\ \cline{1-1} \cline{3-5} 
			4 &  & Flight Idle (on Ground) & Lateral & FFLA \\ \cline{1-1} \cline{3-5} 
			5 &  & Flight Idle (Light on Wheel) & Longitudinal & FLLO \\ \cline{1-1} \cline{3-5} 
			6 &  & Flight Idle (Light on Wheel) & Lateral & FLLA \\ \hline
			7 & \multirow{6}{*}{ON} & Ground Idle & Longitudinal & NILO \\ \cline{1-1} \cline{3-5} 
			8 &  & Ground Idle & Lateral & NILA \\ \cline{1-1} \cline{3-5} 
			9 &  & Flight Idle (on Ground) & Longitudinal & NFLO \\ \cline{1-1} \cline{3-5} 
			10 &  & Flight Idle (on Ground) & Lateral & NFLA \\ \cline{1-1} \cline{3-5} 
			11 &  & Flight Idle (Light on Wheel) & Longitudinal & NLLO \\ \cline{1-1} \cline{3-5} 
			12 &  & Flight Idle (Light on Wheel) & Lateral & NLLA \\ \hline
		\end{tabular}%
	}
\end{table}

Semua data yang didapatkan pada hasil pengukuran \textit{damping ratio} dan vibrasi oleh akselerometer selanjutnya akan diolah dan dianalisis apakah helikopter AS565 MBe Panther hasil modifikasi tersebut berpotensi mengalami fenomena \textit{ground resonance} menggunakan acuan yang terdapat pada tabel \ref{tb:MIL-STD} sehingga didapatkan batas siklus osilasi seperti pada gambar \ref{fig:batas_siklus}. Selanjutnya, data respon frekuensi berupa frekuensi dominan seperti gambar \ref{fig:fft_config2_FILO} akan dimasukkan pada batas siklus osilasi tersebut, yaitu pada gambar \ref{fig:batas_siklus}.

\begin{table}[]
	\centering
	\caption{Tabel acuan MIL-STD-810H-Method-514.8 untuk menghitung batas osilasi yang dimiliki oleh helikopter.}
	\label{tb:MIL-STD}
	\resizebox{0.8\textwidth}{!}{%
		\begin{tabular}{|cccc|}
			\hline
			\multicolumn{1}{|c|}{Materiel} & \multicolumn{1}{c|}{Random Levels} & \multicolumn{1}{c|}{\begin{tabular}[c]{@{}c@{}}Source Frequency ($f_x$)\\ Range (Hz)\end{tabular}} & \begin{tabular}[c]{@{}c@{}}Peak Acceleration ($A_x$) \\ at $f_x$ (Gravity Units (g))\end{tabular} \\ \hline
			\multicolumn{1}{|c|}{\multirow{5}{*}{General}} & \multicolumn{1}{c|}{$W_o = 0.0010 g^2/Hz$} & \multicolumn{1}{c|}{$3 to \leq 10$} & 0.7/(10.70-$f_x$) \\ \cline{2-4} 
			\multicolumn{1}{|c|}{} & \multicolumn{1}{c|}{$W_1 = 0.010 g^2/Hz$} & \multicolumn{1}{c|}{$>10$ to $25$} & 0.10 $f_x$ \\ \cline{2-4} 
			\multicolumn{1}{|c|}{} & \multicolumn{1}{c|}{\multirow{3}{*}{$f_1=500 Hz$}} & \multicolumn{1}{c|}{25 to 40} & 2.5 \\ \cline{3-4} 
			\multicolumn{1}{|c|}{} & \multicolumn{1}{c|}{} & \multicolumn{1}{c|}{40 to 50} & 6.50 - 0.1 $f_x$ \\ \cline{3-4} 
			\multicolumn{1}{|c|}{} & \multicolumn{1}{c|}{} & \multicolumn{1}{c|}{50 to 500} & 1.5 \\ \hline
			\multicolumn{4}{|c|}{Main Rotor Frequencies (Hz)} \\ \hline
			\multicolumn{2}{|c|}{$f_1$ = 1P} & \multicolumn{2}{c|}{Fundamental} \\ \hline
			\multicolumn{2}{|c|}{$f_2$ = n*1P} & \multicolumn{2}{c|}{blade passage (BP)} \\ \hline
			\multicolumn{2}{|c|}{$f_3 = 2f_2$} & \multicolumn{2}{c|}{$2^{nd}$ harmonic} \\ \hline
			\multicolumn{2}{|c|}{$f_4 = 3f_2$} & \multicolumn{2}{c|}{$3^{rd}$ harmonic} \\ \hline
		\end{tabular}%
	}
\end{table}

\section{Perhitungan Matematis}

Pada bagian awal, akan dimasukkan nilai besaran propertis helikopter untuk melengkapi variabel masukan pada persamaan \ref{eq:matriks} menggunakan pendekatan material pada Aluminum 7075-T6. Sehingga didapatkan nilai propertisnya seperti yang terdapat pada tabel \ref{tb:propertis}. Setelah propertis mekanik diberikan, selanjutnya adalah identifikasi melalui grafik seperti yang ditampilkan pada gambar \ref{fig:southwell_coupled_diagram}. Sehingga apabila terdapat solusi dari bagian riil pada nilai eigen yang bernilai positif, maka pada rentang tersebut merupakan frekuensi rotasi rotor yang dapat menyebabkan terjadi \textit{ground resonance}.

\subsection{Validasi Matematis}

Validasi matematis didasarkan pada grafik solusi yang didapatkan dari matriks pada persamaan \ref{eq:EOM}. Aspek validasi menggunakan perpindahan pusat massa pada rotor helikopter pada sumbu-x terhadap data hasil pengukuran helikopter menggunakan akselerometer pada sumbu-x. Sehingga akan didapatkan 2 grafik yang kemudian akan dihitung besarnya perbedaan antara keduanya melalui perhitungan error frekuensi dominan dan \textit{root mean square error} (RMSE). Validasi matematis akan dilakukan terlebih dahulu sebelum analisis perhitungan selanjutnya, agar didapatkan seberapa jauh error yang dimiliki hasil perhitungan matematis terhadap hasil data pengukuran yang didapatkan pada akselerometer (sub bagian \ref{Pengukuran Data Vibrasi Akselerometer}).

\subsection{Perhitungan Matematis Ketidakstabilan kondisi Normal}

Kondisi normal merupakan kondisi saat belum dilakukan modifikasi pada helikopter, atau dengan kata lain belum terdapat penambahan massa pada helikopter. Sehingga pada bagian grafik yang didapatkan dari persamaan \ref{eq:EOM} akan langsung dianalisis rentang frekuensinya. Kemudian akan dilakukan plot grafik saat kondisi frekuensi rotor helikopter berada pada keadaan \textit{ground resonance} dan diluar keadaan \textit{ground resonance}. Sehingga akan didapatkan solusi dalam bentuk grafik yang merepresentasikan kondisi terjadinya \textit{ground resonance} pada komponen helikopter seperti pada gambar \ref{fig:sol_plot_3} dan grafik perpindahan pusat massa rotor pada gambar \ref{fig:xy_3}.

\subsection{Perhitungan Matematis Ketidakstabilan kondisi Modifikasi}

Pada analisis ketidakstabilan hasil modifikasi, besaran massa yang dimasukkan di matriks \textbf{A} pada persamaan \ref{eq:state-space} akan diberikan nilai tambahan massa dengan pertambahan 300 kg, 1000 kg dan 2000 kg sehingga menjadi persamaan \ref{eq:modified} untuk selanjutnya dapat dibandingkan perubahan yang terjadi dari kondisi normal secara matematis untuk kemudian dapat dianalisis pergeseran rentang frekuensi pada helikopter seperti yang dapat dilihat pada gambar \ref{fig:real(modified)_1}.

\section{Pemodelan Simulasi pada Femap}
\label{sec:femap}

\subsection{Skema Pemodelan}



Dalam mencari solusi yang secara tepat merepresentasikan batas siklus osilasi pada perhitungan matematis. Berdasarkan referensi yang telah didapatkan, diperlukan langkah awal membuat skema pemodelan sederhana yang selanjutnya akan membantu untuk mendefinisikan posisi pusat massa dari masing-masing rotor pada helikopter. Skema pemodelan ini berdasarkan pada gambar \ref{tampak_atas.png} untuk menggambar bagian helikopter pada bagian tampak atas. Sedangkan gambar \ref{tampak_samping.png}. Skema gambar menggunakan bantuan \textit{software} desain Inkscape untuk memberikan bentuk \textit{node} pada kerangka yang helikopter yang akan digambar. Gambar \ref{fig:skema_model} ini merupakan skema kerangka helikopter untuk memodelkan helikopter dengan bentuk yang sederhana agar nantinya dapat diolah menggunakan prinsip elemen hingga (\textit{finite element}) pada \textit{software} Femap.

Koordinat pada helikopter didapatkan dengan menggunakan dimensi yang mendekati ukuran sebenarnya dari helikopter. Oleh karena itu pada gambar \ref{fig:skema_model} digunakan bayangan kerangka helikopter yang sebenarnya dari AS565 MBe Panther. Rasio yang dimiliki pada skema gambar tersebut adalah 0.0778m/mm, yang artinya dalam setiap panjang 1mm pada gambar, mewakili 0.0778 meter pada dimensi helikopter yang sebenarnya. Titik 1s', 1s, 3s' dan 3s pada gambar \ref{fig:skema_model} akan dimodelkan dalam kondisi \textit{constraint} (diam) pada arah sumbu-x, sebagai representasi keberadaan ban pada helikopter. 

\begin{figure}[H]
	\centering
	\includegraphics[width=\linewidth]{gambar/rancangan_skema.png}
	\caption{Skema pemodelan helikopter untuk bagian samping, atas, dan bawah.}
	\label{fig:skema_model}
\end{figure}

\subsection{Penentuan Koordinat}

Selanjutnya, pada gambar \ref{fig:skema_model} akan didefinisikan titik-titik yang nantinya akan menjadi \textit{node} pada kerangka helikopter. Titik-titik tersebut akan didefinisikan dalam koordinat kartesian dalam sumbu-x,y, dan z. Titik 2s', 2s dan 3s' akan menjadi titik diberikannya gaya luar untuk simulasi pada FEMAP. Hal ini dikarenakan pada titik tersebut terdapat sensor akselerometer.

Tabel \ref{tb:koordinat} merupakan koordinat yang didefinisikan dengan titik acuan berada pada 1s, sehingga titik 1s didefinsikan dengan koordinat x,y, dan z beruturut-turut, 0, 0, dan 0. Setelah koordinat ditentukan, maka pada tahap selanjutnya adalah melakukan pemodelan pada Femap untuk selanjutnya nanti akan dilakukan analisis menggunakan nilai eigen untuk mencari \textit{mode shape} pada helikopter (gambar \ref{fig:modeshape1} hingga \ref{fig:modeshape10}). Informasi mengenai \textit{mode shape} ini nantinya akan digunakan untuk memberikan gambaran bahwa helikopter akan memiliki respon sedemikian rupa untuk dapat menjelaskan fenomena \textit{ground resonance} yang berpotensi terjadi pada helikopter.

\begin{table}[H]
	\centering
	\caption{Koordinat masing-masing tanda pada titik dalam 3 dimensi.}
	\label{tb:koordinat}
	\resizebox{0.93\textwidth}{!}{%
		\begin{tabular}{|
				>{\columncolor[HTML]{FFCCC9}}c |
				>{\columncolor[HTML]{FFCE93}}c |ccc|
				>{\columncolor[HTML]{FFCCC9}}c |
				>{\columncolor[HTML]{FFCE93}}c |ccc|}
			\hline
			\cellcolor[HTML]{FFCCC9} & \cellcolor[HTML]{FFCE93} & \multicolumn{3}{c|}{Koordinat (m)} & \cellcolor[HTML]{FFCCC9} & \cellcolor[HTML]{FFCE93} & \multicolumn{3}{c|}{Koordinat (m)} \\ \cline{3-5} \cline{8-10} 
			\multirow{-2}{*}{\cellcolor[HTML]{FFCCC9}ID} & \multirow{-2}{*}{\cellcolor[HTML]{FFCE93}Mark} & \multicolumn{1}{c|}{x} & \multicolumn{1}{c|}{y} & z & \multirow{-2}{*}{\cellcolor[HTML]{FFCCC9}ID} & \multirow{-2}{*}{\cellcolor[HTML]{FFCE93}Mark} & \multicolumn{1}{c|}{x} & \multicolumn{1}{c|}{y} & z \\ \hline
			1 & 10s & \multicolumn{1}{c|}{8.4295} & \multicolumn{1}{c|}{0.4676} & 0.3466 & 17 & 4s & \multicolumn{1}{c|}{4.3831} & \multicolumn{1}{c|}{-0.1945} & -0.4085 \\ \hline
			2 & 11s & \multicolumn{1}{c|}{8.8695} & \multicolumn{1}{c|}{0.4806} & 0.3466 & 18 & 4s' & \multicolumn{1}{c|}{4.3831} & \multicolumn{1}{c|}{-0.1945} & 1.1987 \\ \hline
			3 & 12s & \multicolumn{1}{c|}{10.2371} & \multicolumn{1}{c|}{0.6235} & 0.3466 & 19 & 5a & \multicolumn{1}{c|}{1.4013} & \multicolumn{1}{c|}{0.9972} & -0.0374 \\ \hline
			4 & 13s & \multicolumn{1}{c|}{10.2087} & \multicolumn{1}{c|}{1.5276} & 0.3466 & 20 & 5a' & \multicolumn{1}{c|}{1.4013} & \multicolumn{1}{c|}{0.9972} & 0.7295 \\ \hline
			5 & 14s & \multicolumn{1}{c|}{10.9654} & \multicolumn{1}{c|}{3.0033} & 0.3466 & 21 & 5s & \multicolumn{1}{c|}{1.6449} & \multicolumn{1}{c|}{0.9972} & -0.32144 \\ \hline
			6 & 15s & \multicolumn{1}{c|}{11.1567} & \multicolumn{1}{c|}{0.8293} & 0.3466 & 22 & 5s' & \multicolumn{1}{c|}{1.6449} & \multicolumn{1}{c|}{0.9972} & 1.0146 \\ \hline
			7 & 1a & \multicolumn{1}{c|}{3.0017} & \multicolumn{1}{c|}{1.739} & 0.7762 & 23 & 6s & \multicolumn{1}{c|}{2.5934} & \multicolumn{1}{c|}{0.9972} & -0.5704 \\ \hline
			8 & 1a' & \multicolumn{1}{c|}{3.0017} & \multicolumn{1}{c|}{1.739} & -0.0801 & 24 & 7a & \multicolumn{1}{c|}{8.8695} & \multicolumn{1}{c|}{0.4806} & 1.6486 \\ \hline
			9 & 1s & \multicolumn{1}{c|}{0} & \multicolumn{1}{c|}{0} & 0 & 25 & 7a' & \multicolumn{1}{c|}{8.8695} & \multicolumn{1}{c|}{0.4806} & -0.9553 \\ \hline
			10 & 1s' & \multicolumn{1}{c|}{0} & \multicolumn{1}{c|}{0} & 0.6932 & 26 & 7s & \multicolumn{1}{c|}{4.3831} & \multicolumn{1}{c|}{0.9972} & -0.4707 \\ \hline
			11 & 2a' & \multicolumn{1}{c|}{5.3859} & \multicolumn{1}{c|}{1.739} & 0.681 & 27 & 8s & \multicolumn{1}{c|}{2.9028} & \multicolumn{1}{c|}{1.739} & -0.5704 \\ \hline
			12 & 2a & \multicolumn{1}{c|}{5.3859} & \multicolumn{1}{c|}{1.739} & 0.0128 & 28 & 8s' & \multicolumn{1}{c|}{2.9028} & \multicolumn{1}{c|}{1.739} & 1.2081 \\ \hline
			13 & 2s & \multicolumn{1}{c|}{1.019} & \multicolumn{1}{c|}{-0.1945} & -0.4085 & 29 & 9s & \multicolumn{1}{c|}{4.8087} & \multicolumn{1}{c|}{1.739} & -0.4707 \\ \hline
			14 & 2s' & \multicolumn{1}{c|}{1.019} & \multicolumn{1}{c|}{-0.1945} & 1.1987 & 30 & 9s' & \multicolumn{1}{c|}{4.8087} & \multicolumn{1}{c|}{1.739} & 1.1639 \\ \hline
			15 & 3s & \multicolumn{1}{c|}{2.6192} & \multicolumn{1}{c|}{-0.1945} & -0.4085 & 31 & 6s' & \multicolumn{1}{c|}{2.5934} & \multicolumn{1}{c|}{0.9972} & 1.2081 \\ \hline
			16 & 3s' & \multicolumn{1}{c|}{2.6192} & \multicolumn{1}{c|}{-0.1945} & 1.1987 & 32 & 7s' & \multicolumn{1}{c|}{4.3831} & \multicolumn{1}{c|}{0.9972} & 1.1639 \\ \hline
		\end{tabular}%
	}
\end{table}