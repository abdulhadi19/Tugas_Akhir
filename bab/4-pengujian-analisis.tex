\chapter{HASIL DAN PEMBAHASAN}
\label{chap:hasil dan pembahasan}

% Ubah bagian-bagian berikut dengan isi dari pengujian dan analisis

Pada penelitian ini dilakukan 3 aspek pengerjaan dalam menjawab tujuannya, diantaranya adalah data eksperimen yang telah didapatkan dari PTDI. Kemudian validasi matematis menggunakan landasan teori yang telah disesuaikan dengan kondisi pengerjaan penelitian ini, yaitu \textit{ground resonance} helikopter. Kemudian yang terakhir adalah simulasi untuk mendukung hasil data pengukuran dan perhitungan. 

\section{Hasil pengukuran getaran pada FTIS}

Hasil pengukuran data \textit{ground test} dibagi menjadi 2 bagian, seperti yang telah dijelaskan pada bagian metodologi penelitian, bahwa data yang didapatkan merupakan data pengukuran getaran pada FTIS untuk mencari \textit{damping ratio} dan pengukuran data getaran pada akseleromter untuk mendapatkan respon frekuensi dominan oleh helikopter akibat \textit{input} yang diberikan. Berikut ini merupakan grafik yang didapatkan dari masing-masing kondisi yang mengacu pada tabel \ref{tb:variasilanding}.

\begin{figure}[H]
	\centering
	\fbox{\includegraphics[width=0.7\linewidth]{gambar/FTIS_image/All-plot/All_config_1.jpg}}
	\caption{Grafik data hasil pengukuran kondisi 1.}
	\label{fig:condition_1}
\end{figure}

Selanjutnya merupakan grafik dengan keterangan serupa, dimana pada grafik tersebut memiliki keterangan sebagaimana pada tabel berikut:

\begin{table}[]
	\caption{Keterangan pengukuran pada grafik}
	\begin{tabular}{|c|c|c|}
		\hline
		\textit{Lateral cyclic displacement (\%)} & \textit{Longitudinal cyclic displacement} & Pedal displacement (\%)           \\ \hline
		\textit{Roll (deg)}                       & \textit{Pitch (deg)}                      & \textit{Heading (deg)}            \\ \hline
		\textit{Rate of roll (deg/s)}             & \textit{Rate of Pitch (deg/s)}            & \textit{Rate of Yaw (deg/s)}      \\ \hline
		\textit{Acceleration-x ($m/s^2$)}         & \textit{Acceleration-y ($m/s^2$)}         & \textit{Acceleration-z ($m/s^2$)} \\ \hline
	\end{tabular}
\end{table}

Posisi dari masing-masing keterangan berkorelasi dengan posisi pada grafik yang diberikan, contoh: pada bagian awal, dari arah kiri terdapat keterangan "\textbf{\textit{Lateral Cyclic Displacement (\%)}}" maka grafik pada posisi tersebut merepresentasikan "\textbf{\textit{Lateral Cyclic Displacement (\%)}}", begitu seterusnya.

\begin{figure}[H]
	\centering
	\begin{subfigure}{0.49\textwidth}
		\centering
		\fbox{\includegraphics[width=0.9\linewidth]{gambar/FTIS_image/All-plot/All_config_2.jpg}}
		\caption{}
		\label{fig:condition_2}
	\end{subfigure}
	\centering
	\begin{subfigure}{0.49\textwidth}
		\centering
		\fbox{\includegraphics[width=0.9\linewidth]{gambar/FTIS_image/All-plot/All_config_3.jpg}}
		\caption{}
		\label{fig:condition_3}
	\end{subfigure}
	\caption{(a) Grafik data pengukuran pada kondisi-2 (b) Grafik data pengukuran pada kondisi-3.}
\end{figure}

\begin{figure}[H]
	\begin{subfigure}{0.49\textwidth}
		\centering
		\fbox{\includegraphics[width=0.9\linewidth]{gambar/FTIS_image/All-plot/All_config_4.jpg}}
		\caption{}
		\label{fig:condition_4}
	\end{subfigure}
	\centering
	\begin{subfigure}{0.49\textwidth}
		\centering
		\fbox{\includegraphics[width=0.9\linewidth]{gambar/FTIS_image/All-plot/All_config_5.jpg}}
		\caption{}
		\label{fig:condition_5}
	\end{subfigure}
	\caption{(a) Grafik data pengukuran pada kondisi-4 (b) Grafik data pengukuran pada kondisi-5.}
\end{figure}

\begin{figure}[H]
	\begin{subfigure}{0.49\textwidth}
		\centering
		\fbox{\includegraphics[width=0.9\linewidth]{gambar/FTIS_image/All-plot/All_config_6.jpg}}
		\caption{}
		\label{fig:condition_6}
	\end{subfigure}
	\centering
	\begin{subfigure}{0.49\textwidth}
		\centering
		\fbox{\includegraphics[width=0.9\linewidth]{gambar/FTIS_image/All-plot/All_config_7.jpg}}
		\caption{}
		\label{fig:condition_7}
	\end{subfigure}
	\caption{(a) Grafik data pengukuran pada kondisi-6 (b) Grafik data pengukuran pada kondisi-7.}
\end{figure}

\begin{figure}[H]
	\begin{subfigure}{0.49\textwidth}
		\centering
		\fbox{\includegraphics[width=0.9\linewidth]{gambar/FTIS_image/All-plot/All_config_8.jpg}}
		\caption{}
		\label{fig:condition_8}
	\end{subfigure}
	\centering
	\begin{subfigure}{0.49\textwidth}
		\centering
		\fbox{\includegraphics[width=0.9\linewidth]{gambar/FTIS_image/All-plot/All_config_9.jpg}}
		\caption{}
		\label{fig:condition_9}
	\end{subfigure}
	\caption{(a) Grafik data pengukuran pada kondisi-8 (b) Grafik data pengukuran pada kondisi-9.}
\end{figure}

\begin{figure}[H]
	\begin{subfigure}{0.49\textwidth}
		\centering
		\fbox{\includegraphics[width=0.9\linewidth]{gambar/FTIS_image/All-plot/All_config_10.jpg}}
		\caption{}
		\label{fig:condition_10}
	\end{subfigure}
	\centering
	\begin{subfigure}{0.49\textwidth}
		\centering
		\fbox{\includegraphics[width=0.9\linewidth]{gambar/FTIS_image/All-plot/All_config_11.jpg}}
		\caption{}
		\label{fig:condition_11}
	\end{subfigure}
	\caption{(a) Grafik data pengukuran pada kondisi-10 (b) Grafik data pengukuran pada kondisi-11.}
\end{figure}

\begin{table}[H]
	\centering
	\caption{Hasil identifikasi damping ratio rata-rata dari respon helikopter berdasarkan variasi kondisi}
	\label{tb:damp_ratio_table}
	\resizebox{\textwidth}{!}{%
		\begin{tabular}{|c|ccccccccc|}
			\hline
			\multirow{2}{*}{Kondisi} & \multicolumn{9}{c|}{Rata-rata \textit{damping ratio} respon} \\ \cline{2-10} 
			& \multicolumn{1}{c|}{Roll} & \multicolumn{1}{c|}{Pitch} & \multicolumn{1}{c|}{Heading} & \multicolumn{1}{c|}{Rate of Roll} & \multicolumn{1}{c|}{Rate of Pitch} & \multicolumn{1}{c|}{Rate of Yaw} & \multicolumn{1}{c|}{Acceleration-X} & \multicolumn{1}{c|}{Acceleration-Y} & Acceleration-Z \\ \hline
			1 & \multicolumn{1}{c|}{-} & \multicolumn{1}{c|}{-} & \multicolumn{1}{c|}{-} & \multicolumn{1}{c|}{0.05285435} & \multicolumn{1}{c|}{0.036747763} & \multicolumn{1}{c|}{-} & \multicolumn{1}{c|}{-} & \multicolumn{1}{c|}{0.050426643} & - \\ \hline
			2 & \multicolumn{1}{c|}{-} & \multicolumn{1}{c|}{-} & \multicolumn{1}{c|}{-} & \multicolumn{1}{c|}{0.058120539} & \multicolumn{1}{c|}{-} & \multicolumn{1}{c|}{0.045805814} & \multicolumn{1}{c|}{-} & \multicolumn{1}{c|}{0.038387333} & - \\ \hline
			3 & \multicolumn{1}{c|}{-} & \multicolumn{1}{c|}{-} & \multicolumn{1}{c|}{-} & \multicolumn{1}{c|}{0.055225124} & \multicolumn{1}{c|}{0.05507518} & \multicolumn{1}{c|}{0.043144563} & \multicolumn{1}{c|}{-} & \multicolumn{1}{c|}{0.040316101} & - \\ \hline
			4 & \multicolumn{1}{c|}{-} & \multicolumn{1}{c|}{-} & \multicolumn{1}{c|}{-} & \multicolumn{1}{c|}{0.055248847} & \multicolumn{1}{c|}{0.04408422} & \multicolumn{1}{c|}{0.036747763} & \multicolumn{1}{c|}{-} & \multicolumn{1}{c|}{0.062462147} & - \\ \hline
			5 & \multicolumn{1}{c|}{-} & \multicolumn{1}{c|}{-} & \multicolumn{1}{c|}{-} & \multicolumn{1}{c|}{0.064958942} & \multicolumn{1}{c|}{0.054761252} & \multicolumn{1}{c|}{-} & \multicolumn{1}{c|}{-} & \multicolumn{1}{c|}{0.035288539} & - \\ \hline
			6 & \multicolumn{1}{c|}{-} & \multicolumn{1}{c|}{-} & \multicolumn{1}{c|}{-} & \multicolumn{1}{c|}{0.050658456} & \multicolumn{1}{c|}{0.04408422} & \multicolumn{1}{c|}{-} & \multicolumn{1}{c|}{-} & \multicolumn{1}{c|}{0.064120389} & - \\ \hline
			7 & \multicolumn{1}{c|}{-} & \multicolumn{1}{c|}{-} & \multicolumn{1}{c|}{-} & \multicolumn{1}{c|}{0.04732238} & \multicolumn{1}{c|}{0.043670692} & \multicolumn{1}{c|}{0.047821495} & \multicolumn{1}{c|}{-} & \multicolumn{1}{c|}{0.064757723} & - \\ \hline
			8 & \multicolumn{1}{c|}{-} & \multicolumn{1}{c|}{-} & \multicolumn{1}{c|}{-} & \multicolumn{1}{c|}{0.049504081} & \multicolumn{1}{c|}{0.05727551} & \multicolumn{1}{c|}{-} & \multicolumn{1}{c|}{-} & \multicolumn{1}{c|}{0.048365129} & - \\ \hline
			9 & \multicolumn{1}{c|}{-} & \multicolumn{1}{c|}{-} & \multicolumn{1}{c|}{-} & \multicolumn{1}{c|}{-} & \multicolumn{1}{c|}{-} & \multicolumn{1}{c|}{-} & \multicolumn{1}{c|}{-} & \multicolumn{1}{c|}{-} & - \\ \hline
			10 & \multicolumn{1}{c|}{-} & \multicolumn{1}{c|}{-} & \multicolumn{1}{c|}{-} & \multicolumn{1}{c|}{-} & \multicolumn{1}{c|}{-} & \multicolumn{1}{c|}{-} & \multicolumn{1}{c|}{-} & \multicolumn{1}{c|}{-} & - \\ \hline
			11 & \multicolumn{1}{c|}{-} & \multicolumn{1}{c|}{-} & \multicolumn{1}{c|}{-} & \multicolumn{1}{c|}{-} & \multicolumn{1}{c|}{-} & \multicolumn{1}{c|}{-} & \multicolumn{1}{c|}{-} & \multicolumn{1}{c|}{-} & - \\ \hline
			Rata-rata total & \multicolumn{1}{c|}{-} & \multicolumn{1}{c|}{-} & \multicolumn{1}{c|}{-} & \multicolumn{1}{c|}{0.05423659} & \multicolumn{1}{c|}{0.047956977} & \multicolumn{1}{c|}{0.043379909} & \multicolumn{1}{c|}{-} & \multicolumn{1}{c|}{0.050515501} & - \\ \hline
		\end{tabular}%
	}
\end{table}

Pada grafik diatas dibagi menjadi beberapa segmen yang sesuai dengan variasi \textit{input} yang telah diberikan pada tabel \ref{tb:variasi_input} sehingga selanjutnya akan dihitung nilai \textit{damping ratio} dari masing-masing orientasi respon helikopter. \textit{Damping ratio} pada respon helikopter akan dihitung setelah helikopter berhenti memberikan \textit{input}. Sehingga didapatkan tabel \ref{tb:damp_ratio_table} yang memberikan informasi hasil perhitungan \textit{damping ratio} pada helikopter.

Data pengukuran getaran yang didapatkan dari FTIS merupakan data yang diidentifikasi untuk melihat potensi terjadinya \textit{ground resonance} secara visual. Apabila terdapat respon yang meningkat setelah \textit{input} berhenti, maka berdasarkan teori yang telah dijelaskan pada fenomena \textit{ground resonance}. Kondisi tersebut merupakan kondisi awal mula terjadinya \textit{self-excited} yang membuat kerangka helikopter bergetar dengan amplitudo yang semakin besar, sehingga helikopter mengalami kerusakan. Dari variasi \textit{input} yang telah dijelaskan, \textit{input} hanya berasal dari \textit{longitudinal cyclic displacement} dan \textit{lateral cyclic displacement}. Pada gambar \ref{fig:condition_1} hingga \ref{fig:condition_11} dapat dilihat bahwa setiap \textit{input} memiliki respon pada setiap orientasi helikopter, dimulai dari \textit{roll, pitch, heading, rate of roll, rate of pitch, rate of yaw,} percepatan pada arah-X,Y, dan Z. 

Dari tabel \ref{tb:damp_ratio_table} didapatkan informasi bahwa hanya pada orientasi \textit{rate of roll, rate of pitch, rate of yaw,} dan percepatan pada arah sumbu-Y yang dapat diidentifikasi untuk menghitung \textit{damping ratio} helikopter. Hal ini menandakan bahwa helikopter memiliki respon untuk goncangan kanan dan kiri (\textit{rate of roll}) pada tumpuan bannya dengan \textit{damping ratio} rata-rata sebesar 0.054. Respon yang selanjutnya juga memberikan nilai setelah \textit{input} berhenti adalah pada bagian \textit{rate of pitch}, hal ini menandakan bahwa helikopter juga mengalami guncangan ke arah depan dan belakang dengan \textit{damping ratio} rata-rata sebesar 0.0479. Kemudian pada respon \textit{rate of yaw} didapatkan besaran \textit{damping ratio} rata-rata sebesar 0.0433 dan respon percepatan pada arah sumbu-Y dengan \textit{damping ratio} sebesar 0.0505. Hal ini memberikan informasi bahwa helikopter juga memiliki kecenderungan untuk bergerak dengan orientasi arah ke-kanan dan kiri, namun respon tersebut tidak sebanyak dan sebesar pada respon \textit{rate of roll}.

Nilai rata-rata \textit{damping ratio} terbesar diberikan oleh \textit{rate of roll} dan respon percepatan pada arah sumbu-Y. Hal ini berkorelasi dengan orientasi helikopter pada gambar \ref{fig:orientasiheli}. Getaran akan lebih sering terjadi pada arah kanan dan kiri. Secara teori, hal ini disebabkan oleh gaya sentrifugal pada rotor. Disisi lain, orientasi perubahan arah \textit{pitch} rotor juga berubah pada saat bilah pada bilah rotor berada tepat pada bagian kiri dan kanan helikopter.

\begin{figure}[H]
	\centering
	\fbox{\includegraphics[width=0.7\linewidth]{gambar/contoh_grafik_damp_ratio.jpg}}
	\caption{Grafik data pengukuran saat respon masih bergetar pada NLLO kondisi-4.}
	\label{fig:NLLO-4}
\end{figure}

\begin{figure}[h]
	\centering
	\fbox{\includegraphics[width=0.7\linewidth]{gambar/Contoh_TC_Config_3_mark10_FLLO.jpg}}
	\caption{Grafik data pengukuran saat respon teredam dengan cepat pada FLLO kondisi-3.}
	\label{fig:FLLO-3_mark10}
\end{figure}

Hasil perhitungan \textit{damping ratio} pada tabel \ref{tb:damp_ratio_table} didapatkan dengan cara melakukan identifikasi melalui data yang telah di-plot seperti pada gambar \ref{fig:NLLO-4} secara visual, yaitu dengan memberikan tanda pada puncak amplitudo hingga amplitudo terendah, kemudian menghitung besarnya \textit{logarithmic decrement} untuk mendapatkan \textit{damping ratio}. Disisi lain terdapat respon yang langsung teredam setelah \textit{input} berhenti, kondisi ini dapat dilihat pada gambar \ref{fig:FLLO-3_mark10}.

\begin{figure}[h]
	\centering
	\fbox{\includegraphics[width=0.5\linewidth]{gambar/Damping_ratio_plot_ROR.jpg}}
	\caption{Grafik nilai \textit{damping ratio} dari \textit{rate of roll} berdasarkan variasi kondisi.}
	\label{fig:plot_ROR}
\end{figure}

\begin{figure}[H]
	\centering
	\fbox{\includegraphics[width=0.5\linewidth]{gambar/Damping_ratio_plot_ROP.jpg}}
	\caption{Grafik nilai \textit{damping ratio} dari \textit{rate of pitch} berdasarkan variasi kondisi.}
	\label{fig:plot_ROP}
\end{figure}

\begin{figure}[h]
	\centering
	\fbox{\includegraphics[width=0.5\linewidth]{gambar/Damping_ratio_plot_ROY.jpg}}
	\caption{Grafik nilai \textit{damping ratio} dari \textit{rate of yaw} berdasarkan variasi kondisi.}
	\label{fig:plot_ROY}
\end{figure}

\begin{figure}[h]
	\centering
	\fbox{\includegraphics[width=0.5\linewidth]{gambar/Damping_ratio_plot_Acc_y.jpg}}
	\caption{Grafik nilai \textit{damping ratio} dari \textit{acceleration-Y} berdasarkan variasi kondisi.}
	\label{fig:plot_Acc_Y}
\end{figure}

\begin{figure}[H]
	\centering
	\fbox{\includegraphics[width=0.5\linewidth]{gambar/All_damping_ratio.jpg}}
	\caption{Grafik nilai \textit{damping ratio} dari keseluruhan respon helikopter berdasarkan variasi kondisi.}
	\label{fig:plot_All}
\end{figure}

Gambar dari \ref{fig:plot_ROR} hingga \ref{fig:plot_Acc_Y} merupakan plot nilai \textit{damping ratio} berdasarkan masing-masing kondisi untuk orientasi \textit{rate of roll, rate of pitch, rate of yaw,} dan percepatan dalam arah sumbu-Y. Sumbu-x pada gambar tersebut merupakan keterangan pada kondisi ke-berapa terdapat \textit{damping ratio} dengan nilai-nilai yang didapatkan dari perhitungan. Pada gambar \ref{fig:plot_ROR} \textit{damping ratio} tertinggi pada respon \textit{rate of roll} berada pada nilai 0.0912 di kondisi-5 dan terendah pada nilai 0.0291 di kondisi-6. Sedangkan pada gambar \ref{fig:plot_ROP} memberikan informasi nilai \textit{damping ratio} pada \textit{rate of pitch} terbesar pada nilai 0.0572 yang terjadi pada kondisi-8 dan terkecil pada nilai 0.0367 yang terjadi di kondisi-1. Selanjutnya pada gambar \ref{fig:plot_ROY} didapatkan nilai \textit{damping ratio} terbesar pada nilai 0.0478 di kondisi-7 dan terkecil pada nilai 0.0367 di kondisi-4. Kemudian pada arah sumbu-Y, didapatkan nilai \textit{damping ratio} terbesar pada nilai 0.0647 di kondisi-7 dan terkecil pada 0.0352 di kondisi-5.  

Saat pengujian, helikopter memberikan respon getaran yang normal, namun kuantifikasi terhadap kondisi tersebut belum menjelaskan seberapa aman helikopter tersebut dari fenomena \textit{ground resonance}. Perhitungan \textit{damping ratio} setidaknya menjelaskan respon yang dimiliki oleh helikopter terhadap orientasi gerakannya di tanah. Meskipun secara visual, respon helikopter secara jelas tidak menunjukkan adanya potensi terjadinya \textit{ground resonance}. Sehingga identifikasi akan dilanjutkan melalui data hasil pengukuran menggunakan akselerometer.

\section{Hasil pengukuran getaran pada Akselerometer}

Data yang didapatkan dari hasil sensor akselerometer ditunjukkan pada gambar dibawah ini:

\begin{figure}[h]
	\centering
	\fbox{\includegraphics[width=0.7\linewidth]{gambar/raw_config2_ch1.png}}
	\caption{Grafik data pengukuran pada variasi FILO kondisi-2 \textit{channel} 1.}
	\label{fig:raw_config2_FILO}
\end{figure}

\begin{figure}[H]
	\centering
	\fbox{\includegraphics[width=0.7\linewidth]{gambar/fft_config2_ch1.png}}
	\caption{Grafik data hasil FFT pada variasi FILO kondisi-2.}
	\label{fig:fft_config2_FILO}
\end{figure}

Grafik pada gambar \ref{fig:raw_config2_FILO} merupakan grafik data hasil pengukuran yang didapatkan oleh sensor akselerometer pada \textit{channel} 1 dan grafik pada \ref{fig:fft_config2_FILO} merupakan grafik data yang telah diolah menggunakan FFT, sehingga didapatkan nilai dalam domain frekuensi. Data hasil pengukuran menggunakan akselerometer pada gambar diatas akan dibandingkan dengan acuan dari MIL-STD-810H-Method-514.8 (vibrasi). Informasi dari grafik gambar \ref{fig:MIL_STD} didapatkan batas siklus osilasi helikopter. Sehingga didapatkan grafik batas siklus osilasi pada gambar \ref{fig:batas_siklus}.

\begin{figure}[h]
	\centering
	\fbox{\includegraphics[width=0.55\linewidth]{gambar/Oscillation_cycle_limit.jpg}}
	\caption{Grafik batas siklus osilasi dari acuan MIL-STD-810H-Method-514.8 (vibrasi).}
	\label{fig:batas_siklus}
\end{figure}

\begin{table}[H]
	\caption{Tabel perhitungan batas siklus osilasi.}
	\label{tb:batas_siklus}
	\centering
	\begin{tabular}{|ccc|cc|}
		\hline
		\multicolumn{3}{|c|}{Frekuensi (Hz)}                                                       & \multicolumn{2}{c|}{Peak (g)}             \\ \hline
		\multicolumn{1}{|c|}{$f_1$ lower} & \multicolumn{1}{c|}{5.92}  & \multirow{2}{*}{$1p$}       & \multicolumn{1}{c|}{$A_1$ lower} & 0.1464 \\ \cline{1-2} \cline{4-5} 
		\multicolumn{1}{|c|}{$f_1$ upper} & \multicolumn{1}{c|}{6.08}  &                           & \multicolumn{1}{c|}{$A_1$ upper} & 0.1515 \\ \hline
		\multicolumn{1}{|c|}{$f_2$ lower} & \multicolumn{1}{c|}{23.68} & \multirow{2}{*}{$1p*n$}   & \multicolumn{1}{c|}{$A_2$ lower} & 2.368  \\ \cline{1-2} \cline{4-5} 
		\multicolumn{1}{|c|}{$f_2$ upper} & \multicolumn{1}{c|}{24.32} &                           & \multicolumn{1}{c|}{$A_2$ upper} & 2.432  \\ \hline
		\multicolumn{1}{|c|}{$f_3$ lower} & \multicolumn{1}{c|}{47.36} & \multirow{2}{*}{$1p*n*2$} & \multicolumn{1}{c|}{$A_3$ lower} & 1.764 \\ \cline{1-2} \cline{4-5} 
		\multicolumn{1}{|c|}{$f_3$ upper} & \multicolumn{1}{c|}{48.64} &                           & \multicolumn{1}{c|}{$A_3$ upper} & 1.636  \\ \hline
		\multicolumn{1}{|c|}{$f_4$ lower} & \multicolumn{1}{c|}{71.04} & \multirow{2}{*}{$1p*n*3$} & \multicolumn{1}{c|}{$A_4$ lower} & 1.5    \\ \cline{1-2} \cline{4-5} 
		\multicolumn{1}{|c|}{$f_4$ upper} & \multicolumn{1}{c|}{72.96} &                           & \multicolumn{1}{c|}{$A_4$ upper} & 1.5    \\ \hline
	\end{tabular}
\end{table}

Untuk mendapatkan grafik pada gambar \ref{fig:batas_siklus} diperlukan informasi kecepatan rotor helikopter. Diketahui (informasi \textit{flight manual} dari PTDI) nilai kecepatan maksimum dan minimum dari rotor berturut-turut adalah sebesar $365 rpm$ dan $355 rpm$, sehingga didapatkan frekuensi ($f_1$) nya dimulai dari 5.92Hz (minimum) hingga 6.08Hz (maksimum). Perhitungan nilai $f_2$, $f_3$, dan $f_4$ dapat dilihat pada tabel \ref{tb:batas_siklus}. Nilai peak pada tabel tersebut dihitung menggunakan formulasi pada gambar \ref{fig:MIL_STD} yang terletak pada kolom "\textbf{PEAK ACCELERATION ($A_x$) at $f_x$ (GRAVITY UNITS (g))}". Apabila respon sensor akselerometer memiliki nilai frekuensi yang berada diantara batas siklus osilasi dengan amplitudo yang tinggi, maka helikopter memiliki potensi mengalami fenomena \textit{ground resonance} pada frekuensi tersebut. Namun, dari respon yang diberikan oleh helikopter pada semua variasi kondisi dan \textit{input} nilai puncak tertinggi yang dimiliki oleh respon helikopter hanya berada pada nilai 0.143 pada frekuensi 23.68 hingga 24.32 Hz. Sehingga, dari grafik pada gambar \ref{fig:11_FLLO} (kondisi-11 variasi FLLO) tidak ditemukan potensi adanya \textit{ground resonance} pada helikopter. Adapun respon pada kondisi yang lain dengan variasi \textit{input} yang telah ditentukan, responnya memiliki nilai yang lebih kecil dari kondisi pada gambar \ref{fig:11_FLLO}, meskipun respon tersebut berada pada batas siklus osilasi helikopter.

\begin{figure}[h]
	\centering
	\fbox{\includegraphics[width=0.55\linewidth]{gambar/Helicopter_response_GVT_image/Config_11/5_Config_11_FLLO.jpg}}
	\caption{Hasil pengukuran respon frekuensi kondisi-11 pada variasi FLLO pada detik 1800-1850.}
	\label{fig:11_FLLO}
\end{figure}

Grafik pada gambar \ref{fig:11_FLLO} merupakan contoh grafik dengan nilai respon terbesar dibandingkan respon-respon dari kondisi dan variasi lainnya, yaitu dengan amplitudo sebesar 0.143 g-peak pada frekuensi 23.74Hz. Nilai respon yang diberikan pada grafik tersebut merupakan nilai yang didapatkan dari \textit{channel} 1 hingga 6. Data respon yang diberi tanda 'x' merupakan nilai tertinggi dari grafik FFT yang telah didapatkan. Selanjutnya akan coba diidentifikasi bagaimana respon helikopter pada masing-masing \textit{channel} dari kondisi 1 hingga 11.

\begin{figure}[h]
	\centering
	\fbox{\includegraphics[width=0.55\linewidth]{gambar/Plot_per_channel_FTIS/all_channel_1.jpg}}
	\caption{Hasil pengukuran respon frekuensi pada \textit{channel} 1 untuk semua kondisi.}
	\label{fig:channel_1}
\end{figure}

\begin{figure}[h]
	\centering
	\fbox{\includegraphics[width=0.55\linewidth]{gambar/Plot_per_channel_FTIS/all_channel_2.jpg}}
	\caption{Hasil pengukuran respon frekuensi pada \textit{channel} 2 untuk semua kondisi.}
	\label{fig:channel_2}
\end{figure}

\begin{figure}[h]
	\centering
	\fbox{\includegraphics[width=0.55\linewidth]{gambar/Plot_per_channel_FTIS/all_channel_3.jpg}}
	\caption{Hasil pengukuran respon frekuensi pada \textit{channel} 3 untuk semua kondisi.}
	\label{fig:channel_3}
\end{figure}

\begin{figure}[H]
	\centering
	\fbox{\includegraphics[width=0.55\linewidth]{gambar/Plot_per_channel_FTIS/all_channel_4.jpg}}
	\caption{Hasil pengukuran respon frekuensi pada \textit{channel} 4 untuk semua kondisi.}
	\label{fig:channel_4}
\end{figure}

\begin{figure}[H]
	\centering
	\fbox{\includegraphics[width=0.55\linewidth]{gambar/Plot_per_channel_FTIS/all_channel_5.jpg}}
	\caption{Hasil pengukuran respon frekuensi pada \textit{channel} 5 untuk semua kondisi.}
	\label{fig:channel_5}
\end{figure}

\begin{figure}[H]
	\centering
	\fbox{\includegraphics[width=0.55\linewidth]{gambar/Plot_per_channel_FTIS/all_channel_6.jpg}}
	\caption{Hasil pengukuran respon frekuensi pada \textit{channel} 6 untuk semua kondisi.}
	\label{fig:channel_6}
\end{figure}

\begin{table}[H]
	\centering
	\caption{Kuantifikasi banyaknya respon oleh Helikopter pada batas siklus osilasi.}
	\label{tb:identifikasi_pada_batas_siklus}
	\resizebox{\textwidth}{!}{%
		\begin{tabular}{|c|c|c|c|}
			\hline
			Batas siklus & Banyaknya respon (n) & Banyaknya respon (\%) & Total Respon dari (0-80Hz) \\ \hline
			$f_1$ & 0 & 0 & \multirow{4}{*}{2412} \\ \cline{1-3}
			$f_2$ & 383 & 15.87893864 &  \\ \cline{1-3}
			$f_3$ & 103 & 4.270315091 &  \\ \cline{1-3}
			$f_4$ & 15 & 0.621890547 &  \\ \hline
		\end{tabular}%
	}
\end{table}

Grafik pada gambar \ref{fig:channel_1} hingga \ref{fig:channel_6} memberikan informasi terkait respon helikopter dari \textit{channel} 1 hingga 6 untuk semua kondisi. Respon helikopter bertanda 'x' dan diberi warna hijau. Respon tersebut lebih banyak terdapat pada sekitar batas siklus di $f_1$ dan $f_2$. Akan tetapi bila ditinjau secara matematis, tidak terdapat respon helikopter yang berada pada batas siklus $f_1$ hal ini dibuktikan dengan identifikasi nilai kuantitatif respon pada interval $f_1$ seperti yang terdapat pada tabel \ref{tb:identifikasi_pada_batas_siklus}. Dari tabel tersebut, terdapat sebanyak 15.87$\%$ respon helikopter pada batas siklus $f_2$, 4.27$\%$ pada batas siklus $f_3$ dan 0.621$\%$ pada batas siklus $f_4$. Nilai terdekat pada batas siklus $f_1$ berada pada nilai respon frekuensi 5.9166 dan berada pada g-peak 0.0067. Informasi tersebut didapatkan dari data excel yang berjumlah 4539 respon data oleh helikopter seperti yang dapat dilihat pada potongan tabel \ref{tb:list_all_data_sort}. Data yang ditampilkan dan dimasukkan pada batas siklus osilasi hanya merupakan data yang berada pada interval 0 hingga 80Hz. Meskipun pada dasarnya apabila mengacu pada nilai puncak FFT seperti pada gambar \ref{fig:fft_config2_FILO}, didapatkan nilai puncak tertinggi pada rentang frekuensi 700-800Hz.

Identifikasi pada respon frekuensi yang berada di batas siklus osilasi $f_1$ memiliki korelasi yang sangat berkaitan erat dengan fenomena \textit{ground resonance}. Batas siklus osilasi $f_1$ merupakan batas siklus osilasi paling rendah yang berada pada rentang \textit{low frequency}. Berdasarkan penelitian yang telah dikerjakan oleh \cite{Eckert2007AnalyticalAA}, terdapat satu frekuensi pada \textit{lag mode} berupa \textit{low frequency} (frekuensi rendah) yang dapat menyebabkan terjadinya \textit{ground resonance}. Pada rentang frekuensi rendah ini akan dilakukan identifikasi secara matematis sebagai bentuk validasi dari pengukuran data yang telah dilakukan.

\begin{table}[h]
	\centering
	\caption{Tabel respon helikopter pada semua channel dan semua kondisi (potongan).}
	\label{tb:list_all_data_sort}
		\begin{tabular}{|c|c|c|c|}
			\hline
			No & Frekuensi & g-Peak & Channel \\ \hline
			1 & 1.930925 & 0.00063 & ch1 \\ \hline
			2 & 4.869472 & 0.000965 & ch1 \\ \hline
			3 & 7.466667 & 0.000198 & ch1 \\ \hline
			4 & 10.461628 & 0.000213 & ch1 \\ \hline
			5 & 13.358655 & 0.000102 & ch1 \\ \hline
			... & ... & ... & ... \\ \hline
			671 & 5.916678 & 0.006792 & ch2 \\ \hline
			... & ... & ... & ... \\ \hline
			4535 & 1.7132 & 0.01659 & ch6 \\ \hline
			4536 & 1.7683 & 0.01962 & ch6 \\ \hline
			4537 & 1.82582 & 0.02096 & ch6 \\ \hline
			4538 & 23.7427 & 0.02117 & ch6 \\ \hline
			4539 & 746.644 & 0.01644 & ch6 \\ \hline
		\end{tabular}%
\end{table}

\section{Hasil Perhitungan Matematis}

Perhitungan matematis yang dilakukan berdasarkan pada apa yang telah dikerjakan pada \cite{BERGEOT201672}, frekuensi \textit{ground resonance} akan terjadi pada \textit{regressive rotor mode}, yaitu mode yang terjadi pada bilah rotor. Dalam perhitungan tersebut, digunakan matriks dari persamaan \ref{eq:state-space}. Berdasarkan matriks yang telah didapatkan pada persamaan \ref{eq:linearisasi} kemudian disubtitusi pada persamaan \ref{eq:EOM} dan diubah menjadi bentuk \textit{state-space} pada persamaan \ref{eq:state-space_simplified}, maka akan dihitung nilai eigen dari matriks A berikut.

\begin{equation}
	\mathbf{A}=\begin{bmatrix}
	\mathbf{0}& \mathbf{I}\\
	\mathbf{-M}^{-1}\mathbf{K}& \mathbf{-M}^{-1}(\mathbf{C}+\mathbf{G})
	\end{bmatrix}
\end{equation}

Nilai eigen pada matriks A merupakan solusi yang menampilkan besarnya frekuensi pada \textit{fuselage} dan \textit{rotor}. Informasi material pada helikopter akan disubtitusi pada matriks tersebut. Akan tetapi dalam penelitian ini, didapatkan keterbatasan pada spesifikasi propertis helikopter. Sehingga dilakukan pendekatan untuk nilai propertis persamaan diatas menggunakan data kecepatan bilah rotor helikopter, dimana untuk masing-masing propertis adalah sebagai berikut:

\begin{table}[h]
	\centering
	\caption{Pendekatan nilai propertis helikopter.}
	\label{tb:propertis}
	\begin{tabular}{|c|c|c|}
		\hline
		Variabel     & Nilai  	& Besaran \\ \hline
		$L$          & 5.97  	& $m$     \\ \hline
		$k_y$        & 44.100  	& $N/m$   \\ \hline
		$m_y$        & 4.500   	& kg      \\ \hline
		$c_y$        & 441    	& $Ns/m$  \\ \hline
		$m_{\delta}$ & 100.5  	& kg      \\ \hline
		$k_{\delta}$ & 89.028  	& $N/m$   \\ \hline
		$c_{\delta}$ & 106.92 	& $Ns/m$  \\ \hline
	\end{tabular}
\end{table}

Nilai eigen dapat diperhitungkan dari matriks $\mathbf{A}$. Dari hasil tersebut, didapatkan 6 solusi nilai eigen saat kondisi tanpa sistem kopling (interkoneksi antara \textit{input} dan \textit{output}), yaitu saat $\tilde{S}_c = \tilde{S}_d = 0$ (tidak saling berhubungan / \textit{uncoupled}). Sehingga dari kondisi tersebut didapatkan grafik bagian imajiner terhadap kecepatan rotor $\Omega$ dan didapatkan pula bagian rill terhadap kecepatan rotor $\Omega$.

\begin{figure}[H]
	\centering
	\fbox{\includegraphics[width=0.65\linewidth]{gambar/imag_(uncoupled).png}}
	\caption{Plot grafik imajiner terhadap kecepatan rotor $\Omega$ pada kondisi \textit{uncoupled}.}
	\label{fig:imag(uncoupled)}
\end{figure}

Pada grafik \ref{fig:imag(uncoupled)} garis putus-putus berwarna biru merupakan nilai eigen untuk \textit{fuselage} helikopter, sehingga pada bagian imajiner tersebut memberikan besarnya mode dari \textit{fuselage}. Sedangkan garis yang berwarna merah merupakan nilai eigen dari mode rotor helikopter, dimana pada grafik tersebut merepresentasikan \textit{regressive rotor mode} dan \textit{progressive rotor mode}. 

\begin{figure}[H]
	\centering
	\fbox{\includegraphics[width=0.65\linewidth]{gambar/Imag(coupled).jpg}}
	\caption{Plot grafik imajiner terhadap kecepatan rotor $\Omega$ pada kondisi \textit{coupled}.}
	\label{fig:imag(coupled)}
\end{figure}

\begin{figure}[h]
	\centering
	\fbox{\includegraphics[width=0.65\linewidth]{gambar/Real(coupled).jpg}}
	\caption{Plot grafik real terhadap kecepatan rotor $\Omega$ pada kondisi \textit{coupled}.}
	\label{fig:real(coupled)}
\end{figure}

Gambar pada grafik \ref{fig:imag(coupled)} merupakan grafik saat sistem helikopter yang memiliki interkoneksi pada bagian \textit{input} dan \textit{output} nya (\textit{coupled}). $\alpha$ merupakan nilai eigen dari sistem \textit{coupled}. Kondisi saat nilai eigen \textit{coupled} berpotongan dengan garis dari nilai eigennya pada nilai eigen di mode \textit{fuselage} nya akan mengakibatkan terjadinya \textit{ground resonance}. Gambar pada grafik \ref{fig:real(coupled)} memberikan informasi, bahwa pada bagian riil dari nilai eigen sistem \textit{coupled} berada pada nilai positif. Dalam sistem kestabilan, kondisi ini merupakan kondisi yang tidak stabil, sehingga pada kondisi tersebut helikopter akan mengalami getaran yang dapat menyebabkan \textit{ground resonance}. 

\begin{figure}[H]
	\centering
	\fbox{\includegraphics[width=0.65\linewidth]{gambar/find_range_10-11,89.png}}
	\caption{Plot grafik real terhadap kecepatan rotor $\Omega$ pada kondisi \textit{coupled}.}
	\label{fig:resonance_range}
\end{figure}

Gambar \ref{fig:resonance_range} merupakan langkah yang dilakukan untuk mencari respon helikopter yang terdapat pada interval 10.5-11.89Hz (asumsi interval nilai yang berada disekitar frekuensi 11.16). Data respon helikopter berada pada A2 hingga A4540. Langkah ini dilakukan untuk mengonfirmasi apakah dari perhitungan matematis, helikopter memiliki potensi mengalami fenomena \textit{ground resonance} atau tidak. Namun dari hasil tersebut dapat diperhatikan bahwa tidak terdapat respon frekuensi helikopter yang berada pada interval tersebut. Sehingga, secara matematis helikopter juga tidak memiliki potensi terjadinya \textit{ground resonance}.

\section{Potensi \textit{Ground Resonance} pada bentuk Modifikasi Penambahan Massa}

Jika terus dilakukan modifikasi pada helikopter dengan variasi penambahan massa, atau dengan kata lain terjadi perubahan massa pada helikopter. Maka selanjutnya akan dilakukan proses perhitungan untuk memprediksi potensi tersebut terhadap helikopter hasil modifikasi. Asumsikan bahwa pada $m_y$ terdapat penambahan massa baru sebesar $m_m$. Sehingga pada persamaan yang mengandung $m_y$ pada matriks A \ref{eq:state-space} akan menjadi $m'_y$. dengan:

\begin{equation}
	\label{eq:modified}
	m'_y=m_y+m_m
\end{equation}

Sehingga akan dilakukan perhitungan, saat $m_m = 0$ hingga penambahan sampai pada $m_m = 300$ yang menandakan bahwa perubahan pada massanya akan bertambah sebesar 300kg. Selanjutnya akan coba dilihat melalui grafik seperti pada gambar grafik \ref{fig:imag(coupled)} dan \ref{fig:real(coupled)}. Maka akan didapatkan grafik berikut:

\begin{figure}[H]
	\centering
	\fbox{\includegraphics[width=0.65\linewidth]{gambar/imag(modified)_1.jpg}}
	\caption{Plot grafik imajiner terhadap kecepatan rotor $\Omega$ pada kondisi \textit{coupled} sebelum (hitam) dan setelah penambahan massa 300kg (hijau).}
	\label{fig:imag(modified)_1}
\end{figure}

\begin{figure}[H]
	\centering
	\fbox{\includegraphics[width=0.65\linewidth]{gambar/real(modified)_1.jpg}}
	\caption{Plot grafik riil terhadap kecepatan rotor $\Omega$ pada kondisi \textit{coupled} sebelum (hitam) dan setelah penambahan massa 300kg (hijau).}
	\label{fig:real(modified)_1}
\end{figure}

Dari kedua grafik diatas, tidak terdapat perbedaan pada bagian imajiner dan riilnya terhadap sebelum dan sesudah penambahan massa sebesar 300kg. Oleh karena itu selanjutnya akan dicoba penambahan yang lebih berat, yaitu sebesar 1000kg. Sehingga akan didapatkan grafik sebagaimana berikut ini:

\begin{figure}[H]
	\centering
	\fbox{\includegraphics[width=0.65\linewidth]{gambar/imag(modified)_2.jpg}}
	\caption{Plot grafik imajiner terhadap kecepatan rotor $\Omega$ pada kondisi \textit{coupled} sebelum (hitam) dan setelah penambahan massa 1000kg (hijau).}
	\label{fig:imag(modified)_2}
\end{figure}

\begin{figure}[H]
	\centering
	\fbox{\includegraphics[width=0.65\linewidth]{gambar/real(modified)_2.jpg}}
	\caption{Plot grafik riil terhadap kecepatan rotor $\Omega$ pada kondisi \textit{coupled} sebelum (hitam) dan setelah penambahan massa 1000kg (hijau).}
	\label{fig:real(modified)_2}
\end{figure}

Pada grafik diatas, masih tidak terdapat perbedaan bentuk grafik antara sebelum dan sesudah penambahan massa. Kemudian selanjutnya akan dilakukan penambahan massa pada modifikasi sebesar 5000 kg. 

\begin{figure}[H]
	\centering
	\fbox{\includegraphics[width=0.65\linewidth]{gambar/imag(modified)_3.jpg}}
	\caption{Plot grafik imajiner terhadap kecepatan rotor $\Omega$ pada kondisi \textit{coupled} sebelum (hitam) dan setelah penambahan massa 2000kg (hijau).}
	\label{fig:imag(modified)_3}
\end{figure}

\begin{figure}[H]
	\centering
	\fbox{\includegraphics[width=0.65\linewidth]{gambar/real(modified)_3.jpg}}
	\caption{Plot grafik riil terhadap kecepatan rotor $\Omega$ pada kondisi \textit{coupled} sebelum (hitam) dan setelah penambahan massa 2000kg (hijau).}
	\label{fig:real(modified)_3}
\end{figure}

Dari grafik \ref{fig:imag(modified)_3} dan \ref{fig:real(modified)_3} tidak menunjukkan adanya perbedaan yang signifikan sebagai fungsi pada rotor. Kecuali hanya penurunan frekuensi terjadinya fenomena \textit{ground resonance}, yang semula dari 11.16Hz menjadi 11.14Hz. Penambahan ini merupakan penambahan massa yang cukup besar. Pada tabel \ref{tb:propertis} diketahui massa $m_y$ adalah 4500kg, maka penambahan sebesar 2000kg hampir setengah dari massa awal. Dari penambahan tersebut, hanya terjadi penurunan frekuensi \textit{ground resonance} sebesar 0.02Hz. Sehingga dalam hal ini, penambahan massa atau modifikasi pada massa untuk rentang 0 hingga 2000kg pada helikopter maksimal hanya mengalami penurunan frekuensi \textit{ground resonance} sebesar 0.02Hz.