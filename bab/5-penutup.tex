\chapter{PENUTUP}
\label{chap:penutup}

% Ubah bagian-bagian berikut dengan isi dari penutup

\section{Kesimpulan}
\label{sec:kesimpulan}

Berikut merupakan kesimpulan dari hasil yang telah didapatkan:

\begin{enumerate}[nolistsep]

	\item Dari data \textit{ground test}, untuk hasil pengukuran yang didapatkan pada FTIS untuk mencari \textit{damping ratio} secara visual. Tidak ditemukan kondisi yang berpotensi mengalami fenomena \textit{ground resonance}. Helikopter cenderung bergerak ke arah kanan dan kiri (\textit{yaw}) dan ke arah depan dan belakang (\textit{pitch}). Untuk data hasil pengukuran pada akselerometer didapatkan pada rentang $f_1$ (5.09-6.08 Hz) sebanyak 0$\%$. Kemudian pada $f_2$ (23.68-24.32 Hz) sebanyak 15.88$\%$, $f_3$ (47.36-48.64 Hz) sebanyak 4.27$\%$, dan $f_4$ (71.04-72.96 Hz) sebanyak 0.62$\%$. Meskipun terdapat respon frekuensi pada rentang hasil pengukuran pada standar, tidak ditemukan adanya nilai g-peak yang cukup tinggi yang dapat membahayakan helikopter.

	\item Batas siklus osilasi didapatkan menggunakan perhitungan matematis dengan error frekuensi dominan sebesar 2.63$\%$ dan RMSE sebesar 1.7$\%$. Didapatkan untuk kondisi normal adalah berada pada rentang 17.75-20.57 Hz sedangkan untuk kondisi modifikasi dengan penambahan maksimal sebanyak 2000 kg didapatkan rentangnya berada pada 16.25-17.57 Hz dengan jumlah respon frekuensi yang berada pada rentang hasil modifikasi sebanyak 2$\%$.
     
	\item Didapatkan sebanyak 10 \textit{mode shape} untuk masing-masing frekuensi osilasi pada pemodelan dengan error hasil simulasi frekuensi dominan didapatkan sebesar 42.5$\%$ dan RMSE sebesar 5.8$\%$. Didapatkan gerakan yang berkorelasi dengan getaran \textit{ground resonance} pada mode-6, yaitu pada frekuensi 4.13 Hz, serta terdapat sebesar 3.40$\%$ (rentang 4-4.5 Hz) jumlah respon frekuensi.
  
\end{enumerate}

\section{Saran}
\label{chap:saran}

Untuk pengembangan lebih lanjut dengan tujuan yang lebih spesifik pada fenomena \textit{ground resonance} yang terjadi pada helikopter. Berikut ini merupakan beberapa saran yang dapat membangun tujuan yang lebih spesifik tersebut.

\begin{enumerate}[nolistsep]

  \item Diperlukan pengujian pada kerangka helikopter secara langsung dengan menggunakan sensor yang lebih banyak pada titik-titik kerangka helikopter. Hal tersebut untuk mendapatkan informasi yang lebih luas terkait titik-titik yang dapat menyebabkan terjadinya \textit{ground resonance}. Pengujian tersebut seharusnya menggunakan uji tekan dan uji getar menggunakan \textit{shaker} pada kerangka helikopter.

  \item Informasi terkait propertis mekanik pada helikopter sangat dibutuhkan untuk mendapatkan hasil simulasi yang mendekati dengan kondisi eksisting pada helikopter. Sehingga simulasi tersebut dapat lebih dipercaya untuk keperluan pengembangan atau prediksi kondisi pada helikopter yang tidak dimungkinkan untuk dilakukan pengujian langsung.

  \item Pendekatan matematis terhadap pemodelan sederhana helikopter perlu ditinjau kembali secara formulasi, agar ekspresi matematika yang diberikan dapat menjadi acuan validitas terhadap pengujian dan hasil simulasi. Masih terdapat banyak sekali metode yang membangun persamaan matematis pada helikopter, diantaranya adalah kedalaman pemahaman mengenai transformasi \textit{coleman} atau transformasi \textit{multi-blade}, penurunan menggunakan metode \textit{Lagrange}, transformasi binormal, analisis stabilitas linier, dan masih banyak metode matematika yang menjadi dasaran untuk melakukan pendekatan pada helikopter.

\end{enumerate}
