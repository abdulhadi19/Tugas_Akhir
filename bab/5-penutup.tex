\chapter{PENUTUP}
\label{chap:penutup}

% Ubah bagian-bagian berikut dengan isi dari penutup

\section{Kesimpulan}
\label{sec:kesimpulan}

Berikut merupakan hasil yang telah didapatkan:

\begin{enumerate}[nolistsep]

  \item Berdasarkan pengambilan data di tanah saat pengujian pada FTIS dan menggunakan akselerometer dengan variasi kondisi tekanan pada ban serta oleo pesawat, kemudian dengan variasi \textit{input} pada helikopter. Didapatkan bahwa helikopter tidak berpotensi mengalami fenomena \textit{ground resonance} yang dibuktikan dengan grafik pada respon \textit{rate of roll, rate of pitch, rate of yaw} dan percepatan arah sumbu-Y. Selanjutnya berdasarkan data menggunakan akselerometer, tidak ditemukan adanya respon yang berada dalam batas siklus osilasi dengan amplitudo yang tinggi. Bahkan dari semua respon yang dimiliki oleh helikopter, tidak terdapat amplitudo pada frekuensi tertentu yang mencapai setengah dari tinggi puncak batas siklus osilasi.

  \item Helikopter memiliki respon \textit{damping ratio} rata-rata terbesar pada \textit{rate of roll} sebesar 0.054 dan pada orientasi sumbu-Y sebesar 0.050. Hal ini memberikan informasi bahwa helikopter memiliki kecenderungan untuk goncangan kanan dan kiri pada tumpuan ban dan oleonya. Respon frekuensi helikopter tidak terdapat pada batas siklus osilasi $f_1$. Pada hasil perhitungan juga tidak ditemukan adanya respon frekunsi pada rentang 10.5-11.89Hz, yang mana pada perhitungan matematis, rentang tersebut merupakan rentang terjadinya frekunsi \textit{ground resonance}.

  \item Berdasarkan standar MIL-STD-810H-Method-514.8 dan perhitungan matematis, didapatkan batas siklus osilasi pada helikopter berada pada rentang \textit{rmp} natural minimum (5.09Hz) hingga maksimum (6.08Hz) dengan batas puncak pada nilai 0.1464-0.1515 g-peak. Pada rentang tersebut tidak didapatkan satupun respon frekunsi pada helikopter. Namun pada rentang yang lain, yaitu pada rentang 23.68-24.32Hz didapatkan sebanyak 15.87$\%$ dari persentase respon frekuensi batas siklus. Sedangkan pada batas 47.36-48.64Hz didapatkan sebanyak 4.2$\%$ dan pada batas 71.04-72.96Hz sebanyak 0.62$\%$. Pada perhitungan matematis juga tidak ditemukan respon frekuensi pada rentang 10.5-11.89Hz, bahkan jika telah dilakukan modifikasi massa.
  
  \item 

\end{enumerate}

\section{Saran}
\label{chap:saran}

Untuk pengembangan lebih lanjut pada \lipsum[1][1-3] antara lain:

\begin{enumerate}[nolistsep]

  \item Memperbaiki \lipsum[2][1-3]

  \item \lipsum[2][4-6]

  \item \lipsum[2][7-10]

\end{enumerate}
