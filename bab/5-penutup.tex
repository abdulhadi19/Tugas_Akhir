\chapter{PENUTUP}
\label{chap:penutup}

% Ubah bagian-bagian berikut dengan isi dari penutup

\section{Kesimpulan}
\label{sec:kesimpulan}

Berikut merupakan kesimpulan dari hasil yang telah didapatkan:

\begin{enumerate}[nolistsep]

	\item Dari data FTIS, didapatkan bahwa helikopter tidak berpotensi mengalami fenomena \textit{ground resonance} yang dibuktikan dengan grafik pada respon \textit{rate of roll, rate of pitch, rate of yaw} dan percepatan arah sumbu-y (sebagai contoh pada gambar \ref{fig:NLLO-4} dan \ref{fig:FLLO-3_mark10}). Selanjutnya berdasarkan data menggunakan akselerometer, terdapat beberapa respon yang berada dalam batas siklus osilasi. Namun dari respon yang berada pada batas siklus osilasi tersebut tidak ditemukan respon frekuensi yang memiliki amplitudo tinggi, bahkan tidak ditemukan sampai pada separuh nilai batas amplitudo. 

	\item Berdasarkan standar MIL-STD-810H-Method-514.8 dan perhitungan matematis, didapatkan batas siklus osilasi pada helikopter hasil modifikasi berada pada rentang ($f_1=$ 5.09-6.08 Hz), dengan batas puncak pada nilai 0.1464-0.1515 g-peak. Kemudian pada ($f_2=$ 23.68-24.32 Hz), ($f_3=$ 47.36-48.64 Hz), dan ($f_4=$ 71.04-72.96 Hz). Kemudian untuk pendekatan matematis didapatkan rentang batas siklus osilasi pada rentang (17.75-20.57 Hz) dan (16.75-17.57 Hz) pada hasil modifikasi penambahan massa maksimal. Pada hasil perhitungan, aspek penyesuaian yang dipertimbangkan dalam formulasinya melibatkan \textit{lagging motion} pada helikopter dan koefisien kopling yang terkuantifikasi secara matematis. Sehingga menyebabkan perbedaan nilai pada standar yang telah digunakan, yaitu MIL-STD-810H-Method-514.8.
     
	\item Helikopter memiliki respon \textit{damping ratio} rata-rata terkecil pada \textit{rate of pitch} sebesar 0.047 dan \textit{rate of yaw} sebesar 0.043. Hal ini memberikan informasi bahwa helikopter memiliki kecenderungan mengalami orientasi berbelok ke arah kanan dan kiri pada ban dan oleonya serta orientasi \textit{pitch} ke depan dan belakang. Respon frekuensi helikopter tidak terdapat pada batas siklus osilasi $f_1$ (0$\%$). Akan tetapi pada rentang $f_2$ ditemukan sebanyak 313 jumlah respon frekuensi (15.88$\%$). Kemudian pada $f_3$ sebanyak 103 respon frekuensi (4.27$\%$) dan pada $f_4$ sebanyak 15 respon frekuensi (0.62$\%$). Hal ini menunjukkan bahwa pada dasarnya helikopter memiliki kecenderungan memberikan respon frekuensi pada rentang $f_2$. Akan tetapi di lapangan tidak terjadi \textit{ground resonance}, hal tersebut disebabkan oleh g-peak pada respon frekuensi pada helikopter yang kecil. Kemudian tidak ditemukan adanya respon frekuensi yang berada di rentang hasil perhitungan dengan rentang (17.75-20.57 Hz), namun terdapat sebanyak 45 respon frekuensi atau sebanyak $2\%$ untuk modifikasi maksimum pada rentang (16.25-17.57 Hz).
 
	\item Dari hasil pemodelan yang telah dibuat pada Femap dengan menggunakan \textit{finite element} didapatkan 10 \textit{mode shape}. Terdapat satu \textit{mode shape} yang memiliki bentuk gerakan yang serupa dengan fenomena \textit{ground resonance}, \textit{mode shape} tersebut merupakan \textit{mode shape} ke-6 dengan frekuensi sebesar 4.31 Hz. Gerakan tersebut merepresentasikan kondisi saat peredam pada bilah rotor tidak dapat mengkompensasi gerakan tertinggalnya (\textit{lagging motion}) yang mengakibatkan helikopter bergetar belok ke-arah kanan dan kiri dan arah \textit{pitch} depan dan belakang. Terdapat sebanyak 82 frekuensi respon pada frekuensi ini, atau hanya sebesar 3$\%$ dari respon frekuensi total yang berada pada rentang 0 hingga 80 Hz. Hal ini memvalidasi bahwa analisis menggunakan \textit{finite element} menggunakan simulasi pada FEMAP, tidak terdapat potensi terjadinya \textit{ground resonance} pada helikopter AS 565 MBe Panther hasil modifikasi.

\end{enumerate}

\section{Saran}
\label{chap:saran}

Untuk pengembangan lebih lanjut dengan tujuan yang lebih spesifik pada fenomena \textit{ground resonance} yang terjadi pada helikopter. Berikut ini merupakan beberapa saran yang dapat membangun tujuan yang lebih spesifik tersebut.

\begin{enumerate}[nolistsep]

  \item Diperlukan pengujian pada kerangka helikopter secara langsung dengan menggunakan sensor yang lebih banyak pada titik-titik kerangka helikopter. Hal tersebut untuk mendapatkan informasi yang lebih luas terkait titik-titik yang dapat menyebabkan terjadinya \textit{ground resonance}. Pengujian tersebut seharusnya menggunakan uji tekan dan uji getar menggunakan \textit{shaker} pada kerangka helikopter.

  \item Informasi terkait propertis mekanik pada helikopter sangat dibutuhkan untuk mendapatkan hasil simulasi yang mendekati dengan kondisi eksisting pada helikopter. Sehingga simulasi tersebut dapat lebih dipercaya untuk keperluan pengembangan atau prediksi kondisi pada helikopter yang tidak dimungkinkan untuk dilakukan pengujian langsung.

  \item Pendekatan matematis terhadap pemodelan sederhana helikopter perlu ditinjau kembali secara formulasi, agar ekspresi matematika yang diberikan dapat menjadi acuan validitas terhadap pengujian dan hasil simulasi. Masih terdapat banyak sekali metode yang membangun persamaan matematis pada helikopter, diantaranya adalah kedalaman pemahaman mengenai transformasi \textit{coleman} atau transformasi \textit{multi-blade}, penurunan menggunakan metode \textit{Lagrange}, transformasi binormal, analisis stabilitas linier, dan masih banyak metode matematika yang menjadi dasaran untuk melakukan pendekatan pada helikopter.

\end{enumerate}
