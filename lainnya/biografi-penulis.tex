\begin{center}
  \Large
  \textbf{BIOGRAFI PENULIS}
\end{center}

\addcontentsline{toc}{chapter}{BIOGRAFI PENULIS}

\vspace{2ex}

\begin{wrapfigure}{L}{0.3\textwidth}
  \centering
  \vspace{-3ex}
  % Ubah file gambar berikut dengan file foto dari mahasiswa
  \includegraphics[width=0.3\textwidth]{gambar/Abdulhadi.png}
  \vspace{-4ex}
\end{wrapfigure}

% Ubah kalimat berikut dengan biografi dari mahasiswa
\justifying
\name{}, lahir di Makassar pada tanggal 16 September 2000. Sejak umur 3 tahun ikut bersama keluarganya merantau ke tanah Kalimantan dan dibesarkan di Kota Samarinda. Penulis menyelesaikan pendidikan sekolah dasarnya di SD Negeri 004 Pelita Samarinda, kemudian melanjutkan pendidikan menengah pertamanya di MTs Negeri Model Samarinda dan pendidikan menengah tas di MA Negeri Insan Cendekia Paser. Untuk Pendidikan tingkat lanjut, penulis sempat berkuliah di Universitas Hasanuddin pada tahun 2018 di jurusan Fisika. Karena memiliki keinginan untuk berkuliah di tanah Jawa, penulis kembali mengikuti SBMPTN 2019 dan menjatuhkan pilihan pertamanya di Teknik Fisika ITS. Semasa berkuliah, penulis mengikuti beberapa organisasi seperti menjadi Staff Riset dan Teknologi HMTF ITS Kabinet Wani 2021/2022, Staff Dakwah Kreatif JMMI ITS 2019/2020, Kepala Departemen Kominfo Ikatan Alumni Insan Cendekia Paser 2021/2022, dan di beberapa organisasi lainnya. Penulis juga merupakan Asisten Laboratorium Vibrasi dan Akustik, peraih medali Perunggu Gemastik ke-XV pada bidang Karya Tulis Ilmiah TIK, penerima beasiswa Rumah Kepemimpinan Regional-4 Angkatan-X Surabaya, Ketua Himpunan Mahasiswa Teknik Fisika ITS Kabinet Niat 2022/2023 dan penerima Anugerah Wira Adhimukti ITS pada tahun 2022. Penulis mempunya hobi desain grafis dan bermain badminton. Bersamaan dengan kegiatan \textit{Republic of Games} (ROG) 2022 yang diadakan oleh BEM FTIRS, penulis bersama dengan tim Teknik Fisika pernah mendapatkan Juara 1 dalam ajang tersebut. 

Bagi penulis, ITS dan Teknik Fisika memberikan banyak kesempatan dan wadah untuk bisa berkembang. Banyak pelajaran hidup dan pembelajaran mengenai adab menuntut ilmu di dalamnya, meski untuk penulis belum mampu untuk memaknai semua pembelajaran itu dengan baik, penulis terus berusaha untuk memberikan yang terbaik untuk keluarga dan orang-orang disekitarnya.
