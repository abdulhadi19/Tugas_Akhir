\begin{center}
  \Large
  \textbf{KATA PENGANTAR}
\end{center}

\addcontentsline{toc}{chapter}{KATA PENGANTAR}

\vspace{2ex}

Puji syukur kami panjatkan ke hadirat Allah SWT. atas segala rahmat, hidayah, dan karunia-Nya yang senantiasa melimpah dalam setiap langkah perjalanan penyelesaian tugas akhir ini. Melalui buku tugas akhir ini, penulis berharap dapat memberikan kontribusi dan sumbangsih pengetahuan yang bermanfaat bagi perkembangan ilmu pengetahuan dan teknologi, khususnya dalam bidang vibrasi dan struktur \textit{aircraft} helikopter. 

Proses penulisan buku ini tidaklah mudah, namun dengan kesabaran, kerja keras, dan semangat pantang menyerah, penulis berhasil menyelesaikan tugas akhir ini sebagai salah satu syarat dalam menyelesaikan studi S1. Karenanya, penulis ingin mengucapkan terima kasih kepada:

\begin{enumerate}[nolistsep]
	
	\item Umi, Abi, dan keluarga besar yang telah memberikan segala bentuk dukungan kepada penulis. Berkat do'a, kesabaran dan ridho dari keduanya, penulis berhasil menyelesaikan tugas akhir ini. Terima kasih karena telah memberikan kesan baik dan kasih sayang yang tiada berhingga sepanjang masa kepada penulis. 
	
	\item Bapak Dr. Eng Dhany Arifianto, S.T., M. Eng. sebagai Dosen Pembimbing yang telah memberikan arahan yang berharga dalam pengerjaan tugas akhir ini. Dengan penuh kesabaran, beliau telah membimbing penulis dalam menjalankan penelitian ini. 
	
	\item Seluruh Dosen dan Staf pengajar Departemen Teknik Fisika ITS, yang telah memberikan ilmu pengetahuan dan wawasan yang sangat berarti bagi penulis selama masa studi. 
	
	\item Bapak R. Muchamad Bayu Sakti Pratama, ST., MSc-Eng., Ibu Katia M Mahasti S.T., M.T, dan Kak Kezia Grace W. sebagai Pembimbing dan asisten lapangan saat pengambilan data pengukuran di PT. Dirgantara Indonesia yang telah memberikan penulis kesempatan untuk dapat belajar dan meneliti bidang vibrasi pada helikopter.
	
	\item Mas Ahmad Ainun Najib S.T., M.T., Mas Achmadi S.T., M.T., dan Mas Ammar Asyraf S.T., M.T., trio S2 yang senantiasa menjadi teman diskusi tentang banyak hal, terutama dalam cakupan pengerjaan tugas akhir ini.
	
	\item Teman seperjuangan penulis, Tasya, Firda, Diza, Fian, Laila, Ilmi, Opet, Latif, Dias, Tajuddin, Ojha, Mas Pras, Elma dan Adit yang selalu hadir di setiap momen kebahagiaan dan tantangan. Dukungan, semangat, dan tawa yang penulis bagikan bersama telah menjadikan perjalanan ini lebih berwarna dan berarti.
		
\end{enumerate} 

Semoga buku tugas akhir ini dapat memberikan manfaat dan kontribusi bagi pengembangan ilmu pengetahuan dan teknologi. Tentunya, buku ini jauh dari kesempurnaan, oleh karena itu, penulis sangat mengharapkan saran dan kritik yang membangun dari pembaca guna perbaikan dan pengembangan di masa mendatang.

Akhir kata, semoga penelitian ini dapat memberikan inspirasi dan motivasi bagi pembaca seluruhnya. Terima kasih atas segala dukungan dan perhatian yang telah diberikan.

\begin{flushright}
  \begin{tabular}[b]{c}
    \place{}, \MONTH{} \the\year{} \\
    \\
    \\
    \\
    \\
    \name{}
  \end{tabular}
\end{flushright}
