% Atur variabel berikut sesuai namanya

% nama
\newcommand{\name}{Abdul Hadi}
\newcommand{\authorname}{Hadi, Abdul}
\newcommand{\nickname}{Hadi}
\newcommand{\advisor}{Dr. Eng. Dhany Arifianto, S.T., M. Eng.}
%\newcommand{\coadvisor}{Wernher von Braun, S.T., M.T}
\newcommand{\examinerone}{Ir. Wiratno Argo Asmoro, M.Sc.}
\newcommand{\examinertwo}{Dr. Suyanto, S.T., M.T.}
%\newcommand{\examinerthree}{Alan Turing, ST., MT}
\newcommand{\headofdepartment}{Dr. Suyanto, S.T., M.T.}

% identitas
\newcommand{\nrp}{02311940000082}
\newcommand{\advisornip}{19731007 199802 1 001}
%\newcommand{\coadvisornip}{18560710 194301 1 001}
\newcommand{\examineronenip}{19600209 198071 1 001}
\newcommand{\examinertwonip}{19711113 199512 1 002}
%\newcommand{\examinerthreenip}{18560710 194301 1 001}
\newcommand{\headofdepartmentnip}{19711113 199512 1 002}

% judul
\newcommand{\tatitle}{IDENTIFIKASI  BATAS SIKLUS OSILASI  HELIKOPTER AS 565 MBe  PANTHER  TERHADAP  FENOMENA  \textit{GROUND RESONANCE} MENGGUNAKAN  PERHITUNGAN MATEMATIS, \textit{GROUND VIBRATION TEST} DAN \textit{FINITE ELEMENT}.}
\newcommand{\engtatitle}{\emph{IDENTIFICATION OF AS 565 MBe PANTHER HELICOPTER OSCILLATION LIMITS TO GROUND RESONANCE PHENOMENA USING MATHEMATICAL CALCULATIONS, GROUND VIBRATION TEST AND FINITE ELEMENTS.}}

% tempat
\newcommand{\place}{Surabaya}

% jurusan
\newcommand{\studyprogram}{Teknik Fisika}
\newcommand{\engstudyprogram}{Engineering Physics}

% fakultas
\newcommand{\faculty}{Teknologi Industri dan Rekayasa Sistem}
\newcommand{\engfaculty}{Industrial Technology and System}

% singkatan fakultas
\newcommand{\facultyshort}{FTIRS}
\newcommand{\engfacultyshort}{INDSYS}

% departemen
\newcommand{\department}{Teknik Fisika}
\newcommand{\engdepartment}{Engineering Physics}

% kode mata kuliah
\newcommand{\coursecode}{TF 181801}
